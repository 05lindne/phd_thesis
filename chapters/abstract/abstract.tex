%!TEX root = ../../main.tex

\null\vfill

\section*{Abstract}

	Due to their favorable optical properties, silicon-vacancy (SiV) centers have recently emerged as promising candidates for the realization of reliable on-demand \spss. 
	Such non-classical light sources are key to applications in quantum computing, quantum cryptography, and quantum metrology. 
	In the latter \spss are a prerequisite for a quantum-based redefinition of the candela.

	This thesis contributes to the development of \spss with a high applicability in practice through researching \sivs along two main approaches: 
	First the luminescence properties of a large set of \nds containing \sivs are established. 
	This yields a novel strongly inhomogeneous distribution yielding two clusters with regard to the \cwls and the \lw of the \zpl at room temperature.
	One of these clusters is consistently with explained by strain in the diamond lattice, the other might be due to modified \sivs.

	Second, we work towards the goal of developing integrated high-intensity and narrow \lw \spss exploiting the investigated \siv properties.
	Using \pp methods, \sivs are coupled to two different nano-structures.
	By placing a \nd on top of a \vcsel (\VCSEL), we attempt to realize a controllable hybrid-integrated \sps. 
	By coupling \sivs with plasmonic nano-antennas we aim to enhance their \pl intensity. We are able to report significant progress towards this goal.

	Our contributions add momentum to the development of integrated, high-intensity, narrow \lw \spss, to the development of novel calibration standards and ultimately to the universal adoption of the quantum candela.



% \end{abstract}

\vfill

\newpage

\null\vfill

\section*{Zusammenfassung}

	Auf Grund günstiger optischer Eigenschaften sind Silizium Farbzentren (SiV) vielversprechende Kandidaten für die Realisierung von Einzelphotonenquellen. Solche nicht-klassischen Lichtquellen für Quantencomputer, Quantenverschlüsselung und Quantenmetrologie essentiell. Für letztere bilden Einzelphotonenquellen eine Schlüsseltechnologie auf dem Weg zu eine quanten-basierte Neudefinition der Basiseinheit Candela.

	Diese Arbeit verfolgt zwei Ansätze um die Entwicklung von Einzelphotonenquellen mit praktischer Bedeutung voranzutreiben: 
	Im ersten Ansatz werden die Lumineszenzeigenschaften einer großen Menge von Nanodiamanten welche SiV Zentren enthalten etabliert. 
	Im Zuge dessen wird eine bisher unbekannte, stark inhomogene Verteilung entdeckt, die in zwei Cluster bezüglich der Zentralwellenlänge und Linienbreite der Nullphononenlinie bei Raumtemperatur unterteilt ist.
	Eines dieser Cluster wird mit Spannungen im Diamantgitter erklärt, das andere könnte aus modifizierten SiV Zentren bestehen.

	Im zweiten Ansatz arbeiten wir an der Entwicklung von integrierten Einzelphotonenquellen mit hoher Intensität und geringer Linienbreite basierend auf den untersuchten SiV Eigenschaften. Mittels “pick-and-place“ Methoden werden SiV Zentren an verschiedene Nanostrukturen gekoppelt. 
	Durch das platzieren auf einem Oberflächenemitter (VCSEL) versuchen wir eine kontrollierbare hybrid- integrierte Einzelphotonenquelle zu realisieren. 
	Durch die Kombination von Nanodiamanten mit plasmonischen Nanoantennen versuchen wir die Fluoreszenzintensität von enthaltenen SiVs zu verbessern. 
	Es wurden signifikante Fortschritte in Bezug auf beide Ziele erreicht.

	Unsere Arbeit trägt zur Entwicklung integrierter Einzelphotonenquellen mit hoher Intensität und geringer Linienbreite bei. 
	Dadurch leisten wir einen Beitrag zu Entwicklung neuer Kalibrierungsstandards und damit zur Einführung des Quantencandela. 

\vfill
