%!TEX root = ../../main.tex
\null\vfill
% \begin{abstract}


\section*{Abstract}

	Due to their favorable optical properties, \sivc{}s have recently emerged as promising candidates for the realization of reliable on-demand \spss. Such non-classical light sources are key to a plethora of applications, notably the proper calibration of detectors designed to operate in the single-photon counting regime.

	This thesis contributes to the development of \spss through researching \sivs along two main approaches: First the luminescence properties of a large set of \nds containing \sivs are established. This yielded a novel distribution of select emitter properties and adds to our understanding of \sivs, improving our abilities to deploy \sivs as constituents of integrated \sps.

	Second, we attempt to realize two such a devices using \nds hosting \sivs. Using \pp methods, we relocate individual \nds to two different nano structures, thus achieving a coupling between \sivs in the \nd and the nano structure. By placing a \nd ontop of a \vcsel, we attempt to realize a controllable integrated \sps. By combining \nds hosting several \sivs with plasmonic nanoantennas we aim to enhance the emission properties of \sivs. Unfortunately, a completion of our devices was not possible due to significant experimental challenges.

	However, we believe that our contributions add momentum to the development of \sps, to the development of novel calibration standards and photon detectors and ultimately, to the universal adoption of the quantum candela.



% \end{abstract}

\vfill

\newpage

\null\vfill

\section*{Zusammenfassung}

	Due to their favorable optical properties, \sivc{}s have recently emerged as promising candidates for the realization of reliable on-demand \spss. Such non-classical light sources are key to a plethora of applications, notably the proper calibration of detectors designed to operate in the single-photon counting regime.

	This thesis contributes to the development of \spss through researching \sivs along two main approaches: First the luminescence properties of a large set of \nds containing \sivs are established. This yielded a novel distribution of select emitter properties and adds to our understanding of \sivs, improving our abilities to deploy \sivs as constituents of integrated \sps.

	Second, we attempt to realize two such a devices using \nds hosting \sivs. Using \pp methods, we relocate individual \nds to two different nano structures, thus achieving a coupling between \sivs in the \nd and the nano structure. By placing a \nd ontop of a \vcsel, we attempt to realize a controllable integrated \sps. By combining \nds hosting several \sivs with plasmonic nanoantennas we aim to enhance the emission properties of \sivs. Unfortunately, a completion of our devices was not possible due to significant experimental challenges.

	However, we believe that our contributions add momentum to the development of \sps, to the development of novel calibration standards and photon detectors and ultimately, to the universal adoption of the quantum candela.


\vfill
