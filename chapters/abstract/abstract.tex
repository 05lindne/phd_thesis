%!TEX root = ../../main.tex
\null\vfill
% \begin{abstract}


\section*{Abstract}

	Due to their favorable optical properties, silicon-vacancy (SiV) centers have recently emerged as promising candidates for the realization of reliable on-demand \spss. Such non-classical light sources are key to various applications, not only in quantum computing and quantum cryptography, but also in quantum metrology. In the latter \spss are a prerequisite for a quantum-based redefinition of the candela.
	\\
	This thesis contributes to the development of \spss with a high applicability in practice through researching \sivs along two main approaches: First the luminescence properties of a large set of \nds containing \sivs are established. This yields a novel distribution of select emitter properties and adds to our understanding of \sivs in \nds, improving our abilities to deploy \sivs as constituents of integrated \spss.
	\\
	Second, we work towards the goal of developing integrated high-intensity and narrow \lw \spss using \nds hosting \sivs. Using \pp methods, we relocate individual \nds to two different nano structures, thus achieving a coupling between \sivs in the \nd and the nano structure. By placing a \nd on top of a \vcsel (\VCSEL), we attempt to realize a controllable hybrid-integrated \sps. By combining \nds hosting \sivs with plasmonic nano-antennas we aim to enhance the \pl intensity of \sivs. We are able to report significant progress towards this goal, however significant experimental challenges remain to be addressed.
	\\
	Our contributions add momentum to the development of integrated, high-intensity, narrow \lw \spss, to the development of novel calibration standards and photon detectors and ultimately, to the universal adoption of the quantum candela.



% \end{abstract}

\vfill

\newpage

\null\vfill

\section*{Zusammenfassung}
	\todo{translate abstract to german}

	Due to their favorable optical properties, silicon-vacancy (SiV) centers have recently emerged as promising candidates for the realization of reliable on-demand \spss. Such non-classical light sources are key to various applications, not only in quantum computing and quantum cryptography, but also in quantum metrology. In the latter \spss are a prerequisite for a quantum-based redefinition of the candela.
	\\
	This thesis contributes to the development of \spss with a high applicability in practice through researching \sivs along two main approaches: First the luminescence properties of a large set of \nds containing \sivs are established. This yields a novel distribution of select emitter properties and adds to our understanding of \sivs in \nds, improving our abilities to deploy \sivs as constituents of integrated \spss.
	\\
	Second, we work towards the goal of developing integrated high-intensity and narrow \lw \spss using \nds hosting \sivs. Using \pp methods, we relocate individual \nds to two different nano structures, thus achieving a coupling between \sivs in the \nd and the nano structure. By placing a \nd on top of a \vcsel (\VCSEL), we attempt to realize a controllable hybrid-integrated \sps. By combining \nds hosting \sivs with plasmonic nano-antennas we aim to enhance the \pl intensity of \sivs. We are able to report significant progress towards this goal, however significant experimental challenges remain to be addressed.
	\\
	Our contributions add momentum to the development of integrated, high-intensity, narrow \lw \spss, to the development of novel calibration standards and photon detectors and ultimately, to the universal adoption of the quantum candela.

\vfill
