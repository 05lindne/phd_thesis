%!TEX root = ../../main.tex


 \chapter*{Summary and Conclusions}	\label{ch::conclusion}
 \addcontentsline{toc}{chapter}{Summary and Conclusions}
   % short summary of what happened in the thesis

   % important stuff from siv paper

   In conclusion we found a strongly inhomogeneous distribution of \siv spectra in nanodiamonds.
   We group the \zpls into two groups:
   \Hl consists of \ZPLs exhibiting a narrow \lw from below \SI{1}{nm} up to \SI{4}{nm} and a broad distribution of \cwl between \SIlist{710;840}{nm}.
   Compared to that, \vl comprises \ZPLs with a broad \lw between just below \SIlist{5; 17}{nm} and \cwl ranging from \SIrange{730}{742}{nm}.
   This data is consistent with previously measured SiV spectra \cite{Benedikter2017a,Neu2012}.
   For comparison, an \siv in unstrained bulk diamond exhibits a \lw of \SIrange{4}{5}{nm} at a center wavelength of \SI{737.2}{nm}\cite{Arend2016a,Dietrich2014}.
   We show that both the observed blue-shift and the observed red-shift of \vl are consistently explained by strain in the diamond lattice.
   Further, we suggest, that \hl might be comprised of modified \sivs, the structure of which is currently unclear.
   \\
   We investigated the SiV sidebands:
   In \vl we found one prominent peak at a shift of \SI{42}{meV}, which corresponds to a well-known feature assigned to non-localized lattice vibrations \cite{Larkins1971,Sternschulte1994}.
   In \hl we see an accumulation of peaks, at around \SIlist{43;64;150;175}{meV}, which are consistent with sideband peaks reported in \cite{Sternschulte1994,Zaitsev2001,Neu2011}.
   \\
   We further reported photon autocorrelation measurements which verified the existence of single \sivs both in \hl and \vl.
   Investigating the time trace of the \siv \pl, we found that some \sivs exhibit fluorescence intermittence with on times between several microseconds up to \SI{41}{s}.
   Furthermore, we see an exponential distribution of bright time intervals and a log-normal distribution of dark time intervals, consistent with research on single molecules \cite{Wong2013}.
   In terms of photostability, there is a big difference between \vl and \hl:
   All but one emitters in \hl exhibit blinking, where only one of the emitters in \vl exhibits blinking.


   % important facts from coupling chapter

   To effectivly quantify the emission enhancement provided by the plasmonic double bowtie antenna, a single \siv is necessary.
   A correct measure for the emission enhancement is the saturation count rate since it is proportional to the inverse of the emitter's lifetime.
   Hence, if there are two or more emitters present, photons of the individual emitters are emitted randomly, which renders a correct saturation measurement impossible.
   However, finding \sivs in \nds which fulfill both spectroscopic (\gtz $\approx$0, saturation, narrow \ZPL spectrum) and technical (size, isolation of \nds) constraints is an extremely time-consuming process. It is a search for the needle in the haystack.
   \\
   We investigated different kinds of \nds in the search of \nds exhibiting optimal spectroscopic and technical parameters.
   We were able to fulfill the size requirements posed by the \pp process and antenna design by producing different patches of different sizes of \nds and took the ones which were best suited.
   We also developped a good isolation of the \nds on the substrate by treating the \ir substrate with Piranha etch and tuning the amount of diamond solution drop-casted onto the substrate.
   This leaves us with the need of a higher propability of exactly one \siv per \nd.
   Parameters which have an impact on the quantity of \sivs per \nd are the initial \siv density in the starting material and the \nd size.
   Once the time constraint of finding a single \siv in a \nd is overcome, we can apply the extensive methods and knowledge gained by the reported procedures to couple a single \siv to a plasmonic bowtie antenna.
   \\
   To our knowledge, our experiments were the first attempts of coupling \sivs to plasmonic bowtie antennas.
   The extraordinarily precise correlation of the theoretically predicted and the experimentally recorded spectrum of an ensemble of \sivs in a \nd make this process a promising candidate for future applications.


   % establishing connection to introduction and quantum candela

   adding momentum towards the adoption of the quantum candela \dots
