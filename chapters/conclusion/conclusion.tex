%!TEX root = ../../main.tex


 \chapter*{Summary and Conclusions}	\label{ch::conclusion}
 \addcontentsline{toc}{chapter}{Summary and Conclusions}
   % short summary of what happened in the thesis

   In this thesis we explore selected properties of the \si \cc (\siv) hosted in nano-sized diamond grains (\nds). Studying the \sivc in diamond is motivated by the fact that this particular \cc has surfaced as a promising candidate for realizing reliable on-demand \spss. Such non-classical light sources are key for the proper calibration of detectors and other instruments designed to operate reliably in the single-photon counting regime. This in turn is expected to drive a shift towards quantum based radiometric SI units, such as the quantum candela, which in the long run is expected to foster improvements in our ability to work with individual photons in a wide range of practical applications.

   The \siv has the potential to add considerable momentum to this cause. It is are very efficient and stable narrow \lw emitter, emitting single photons with high intensity. Conveniently, \sivs operate at room temperature under normal pressure and hence do not require extremely sophisticated experimental setups. As an alternative to hosting \sivs in bulk diamond, they can be implanted in nano-sized diamond grains offering increased collection efficiency. Grains containing individual \sivs can be identified and preselected according to their properties. As a result individual emitters can be made mobile using \pp techniques. In our work, we demonstrate that this ability unlocks applications requiring select single emitters such as coupling to antennas or the use as hybrid integrated \spss in conjunction with vertical-cavity surface-emitting laser (\VCSEL).

   To enable our research, we synthesize \nds containing \sivs using a variety of different techniques. \Cvd, \hpht synthesis as well as wet-milling methods are used to produce a sizable set of samples. To investigate them, i.e\ to study the optical properties of embedded \sivs we rely on optical excitation. In particular, confocal microscopy is used to collect emitted \fl single photons. An attached spectrometer or a \HBT setup offer further insights into the properties of the \fl emission.

   Our work can roughly be subdivided into two larger explorations. The first, reported in \cref{ch::distribution}, revolves around charting the luminescence properties of a large set of emitters, allowing us to establish distributions of selected \siv properties. To our knowledge, our efforts result in the largest coherent examination of \sivs to date. In contrast to that the second exploration focuses on individual \nds and our attempts to relocate them to different nano structures in order to achieve a coupling between \sivs and the host structure. \cref{ch::coupling} details our efforts to realize a controllable integrated \sps by placing a \nd containing \sivs ontop of \vcsel (\VCSEL). Furthermore, we combine \nds with plasmonic nano-antennas in order to enhance the emission properties of the contained \sivs. A detailed summary of the results obtained in both explorations follows below.

   % \section{\siv paper}

   With regards to the investigation of large sets of emitters we find a previously unknown strongly inhomogeneous distribution of \siv spectra in \nds. The distribution in particular suggest a partitioning of emitters according to the \wl and \lw of their \zpls (\ZPL) into two groups, a vertical (\vl) and a horizontal (\hl) group.

   \Hl consists of \ZPLs exhibiting a narrow \lw from below \SI{1}{nm} up to \SI{4}{nm} and a broad distribution of \cwl between \SIlist{710;840}{nm}.
   Compared to that, \vl is comprised of \ZPLs with a broad \lw between just below \SIlist{5; 17}{nm} and \cwl ranging from \SIrange{730}{742}{nm}.
   This data is consistent with previously measured SiV spectra \cite{Benedikter2017a,Neu2012}.
   For comparison, an \siv in unstrained bulk diamond exhibits a \lw of \SIrange{4}{5}{nm} at a \cwl of \SI{737.2}{nm}\cite{Arend2016a,Dietrich2014}.
   We show that both the observed blue-shift and the observed red-shift of \vl are consistently explained by strain in the diamond lattice.
   Further, we conjecture, that \hl might be comprised of modified \sivs, the structure of which is currently unclear.

   In addition to the \zpl, we also investigate the \psb of the available emitters. In \vl we find one prominent peak at a shift of \SI{42}{meV}, which corresponds to a well-known feature assigned to non-localized lattice vibrations \cite{Larkins1971,Sternschulte1994}.
   In \hl we see an accumulation of peaks, at around \SIlist{43;64;150;175}{meV}, which are consistent with sideband peaks reported in \cite{Sternschulte1994,Zaitsev2001,Neu2011}.

   We further report photon autocorrelation measurements asserting the existence of single \sivs acting as \sps both in \hl and \vl.
   Investigating the time trace of \siv \pl, we find that some \sivs exhibit fluorescence intermittence, also known as blinking, with on-times ranging from several microseconds up to \SI{41}{s}. All but one emitters in \hl exhibit blinking, where only one of the emitters in \vl exhibits blinking.

   Furthermore, we establish for the first time an exponential distribution for bright-time intervals and a log-normal distribution for dark-time intervals, consistent with research on single molecules \cite{Wong2013}.

   % \section{Coupling}

   After reporting the results of charting the properties of large sets of \nds and their contained \sivs, we move on to working with individual \nds.
   The ability to examine \sivs individually opens up the possibility to preselect emitters according to desired spectroscopic parameters, such as narrow \lws, high count-rates and single photon emission.

   Once such suitable candidates are identified, they can be moved between different substrates with precision using \pp methods. Although these methods are difficult to execute, \sivs may be relocated and coupled to photonic structures where their extraordinary properties can be exploited to create \sps. In the scope of this thesis, \nds including suitable \sivs were identified and coupled to two different kinds of structures: \Vcsels (\VCSELs) and plasmonic nano-antennas.

   % \subsection{\Vcsels}

   Red AlGaInP-based oxide-confined \VCSELs are perfect candidates for the excitation of \sivs in a hybrid integrated \sps. Their physical size and the fact, that their output beam is perpendicular to the substrate implies that candidate \nds containing \sivs can simply be placed on-top of the light-emitting region of the structure relying on \pp methods. Then, the \VCSELs output laser can be used to optically excite \sivs, steering emission of the \siv indirectly via the operation of the \VCSEL. Through the use of suitable optical filters allowing only \siv \fl to emerge, a controlled \sps can be realized in principle. This system is interesting in the context of metrological applications, as it constitutes a promising building block for a portable device ready to calibrate single photon detectors.

   As a first step we characterize the behavior of the \VCSEL itself, establishing a lasing \wl of \SI{655}{\nm} for operation currents of \SI{1.5}{\mA} and \SI{3}{\mA}. At the same time a maximum output power of was established $\approx \SI{0.6}{\mW}$. Thus, the \VCSEL  is suitable for exiting \sivs.

   Next we placed a \nd containing \sivs on-top of the output region of the \VCSEL using \pp methods. We compare spectra of the \nd before and after the relocation process and unfortunately find that the major \zpl feature at \SI{746.0}{nm} disappeared with a formerly minor peak remaining as the dominant feature. Damage to the \cc due to exposure to electron radiation during the \pp process seems the most likely explanation for this development.

   With one significant peak still remaining in the spectrum we decide to proceed to use the \Vcsel to exite the \sivs contained in the \nd.
   Since the position of the \nd on-top of the \VCSEL is known, we measure the spectrum of light originating from this position. A second spectrum is obtained from an identical \VCSEL measuring the corresponding position without a \nd present. Comparing the two spectra, no meaningful difference can be found. In particular, a lack of any distinct lines in the spectrum of the \VCSEL  with \nd attributable to \siv emission is striking. We believe that this observation is caused by significant \VCSEL sideband contributions in the \wl range relevant for \siv fluorescence.

   Although our initial attempt to realize a hybrid-integrated \sps by coupling \sivs to a \VCSEL is thwarted by large \sb emissions of the latter, a clear direction for further improvements was identified. In particular it seems promising to explore the idea of putting an additional gold layer on-top of the \VCSEL. If well-engineered it could be used a mirror to suppress the \VCSEL \sb in the relevant \siv emission regime.

   % \subsection{Antennas}

   After attempting to couple individual \nds and their emitters to \VCSELs, we proceed to explore the possibilities of combining \sivs with plasmonic nano-antennas. Nano-antennas are very recent nano-scale devices designed to efficiently convert freely propagating optical radiation into localized energy and vice versa. Here we use gold double bowtie antennas on a gold substrate. This design allows the creation of focused electromagnetic field at the gap in the center of the antenna, a so-called hot-spot. Placing emitters in such a hot-spot is known to increase their spontaneous emission rate provided their emission frequency is close to the resonance frequency of the antenna. This is known as Purcell effect which can be used to improve the emission of emitters \cite{nancy::86}.

   In our work we aim to place a \nd in the hot-spot of a double bowtie antenna in order to enhance the emission of the hosted \sivs. A necessary condition for enhancement is that the frequencies of the \siv emission and the resonance frequency of the antenna match. Furthermore, the gap of the antenna must be large enough to accommodate \nds. Feasible physical design parameters for the antenna meeting these requirements were determined using finite time difference domain (FDTD) simulations. After suitable antennas were fabricated, their resonance frequencies were verified experimentally.

   At this point, suitable \nds containing exactly one \siv each can be placed in the center of the antenna to study the integrated system. The term suitable is used to summarize both desirable spectroscopic properties such as narrow-bandwidth saturated single-photon emission as well as technical requirements such as \nd size and sufficient degree of isolation on the surface to enable \pp. Naturally, the odds of identifying and addressing a \nd fulfilling all these criteria simultaneously are small. As a result identifying a perfect candidate for coupling is prohibitively time-consuming.

	 To mitigate this difficulty we decided to relax the condition of exactly one \siv per \nd and initiate our exploratory work with \nds containing several, potentially many active \sivs. Relying on \gtz measurements we identify two interesting classes of \nds. The first class consists of \nds containing large ensembles of \sivs acting as coherent emitters. The \fl light received from large ensemble of emitters is mainly coherent, leading to a flat response in the \gtz function. The second class of \nds we investigate features \nds hosting multiple \sivs. As a result relevant \gtz measurements report weak but discernible anti-bunching dips. Both classes have in common that relevant \nd specimen are significantly easier to obtain than \nds containing singleton \sivs. Thus \nds containing ensembles of \sivs as well as \nds containing few \sivs are both valid starting points for our exploration. To our knowledge, our experiments were the first attempts of coupling \sivs to plasmonic bowtie antennas.

   % \subsubsection{Ensembles}

   We begin by selecting a suitable \nd containing a large ensemble of \sivs and measure its spectrum in isolation. Using FDTD simulations we obtain the predicted spectrum of the antenna by itself without a \nd present. A convolution of both spectra yields a prediction of what the spectrum of the combined system consisting of \nd and antenna should look like. Indeed, our measurements of the integrated systems are congruent with the prediction asserting successful coupling facilitated by the \pp method. The remarkable agreement of the theoretically predicted and the experimentally recorded spectra for an ensemble of \sivs in a \nd make our simulation assisted approach very useful and promising candidate for future explorations of antenna-emitter coupling.

   We remark at this point, that a meaningful quantification of the Purcell emission enhancement induced by the coupling can in principle be achieved via intensity saturation measurements. By measuring the saturation intensities before and after insertion into the antenna and accounting for effects related to the polarization of emitters, the magnitude of the Purcell enhancement can be determined. Unfortunately, these methods are reserved for single emitters and do not apply for ensembles of \sivs. Thus at this point we have no method to determine the Purcell enhancement ensembles of \sivs experience.

   % \subsubsection{small number}

   Next we report our attempts to couple \nds hosting a limited number of \sivs to plasmonic gold double bowtie antennas. Suitable \nds show an anti-bunching dip in the \gtf in addition to count-rate saturation. This can be regarded an intermediate step towards using \nds containing singleton \sivs.

   Analogous to the process of coupling \nds with large ensembles of \sivs, we begin by recording the spectrum of the \nd in isolation. By folding the obtained spectrum with the FTDT simulated spectrum of an antenna without \nd we once again obtain a prediction of what the spectrum of the integrated system should look like. Next we relocate the \nd in question to the center of the antenna using the \pp method. Having assembled the integrated system, we record its spectrum to compare it with the simulation-assisted prediction. Unfortunately, this time the results disagreed significantly. This surprising event prompted us to verify the experimental setup. Since no misalignment could be detected we proceeded to redo the spectrum in an attempt to assert the spectrum obtained earlier. However, the \sivs hosted by the \nd seemed to have bleached, permanently invalidating the \nd in the antenna. The obtained spectrum showed only the expected result for the antenna in isolation.

   It is likely that continued application of energy from the laser triggered this effect as earlier independent measurements established that \sivs exhibit an increased likelihood of bleaching after being exposed to electron radiation. Thus we conjecture that exposure to electron radiation during the \pp left the \sivs in an unstable state susceptible to bleaching.

   Even though the \nd in the antenna was invalidated for further measurements, the spectrum that was obtained in the initial measurement remains to be discussed further.

   Given the lack of agreement between measurement and simulation-assisted prediction we must conclude that the spectrum of the \nd was modified during the \pp process. While it is not possible to pinpoint exactly which circumstance caused the modification, several effects could influence the observed spectra. These include damage through electron radiation as well as additional fluorescing surface contaminations the \nd might have picked up during the \pp process. The fact that a significant sideband signals were recorded supports the latter conjecture. It also seems logical to assume that \nds containing only a few \sivs are less resilient to radiation damage than \nds containing large ensembles of \sivs.
   Yet another property of \sivs which should not be neglected is their dipole orientation interacting with the antenna. Conducting dedicated FDTD simulations with a focus on different dipole orientations indicated a dramatic effect on the resulting spectra. Therefore, future experiments aiming to investigate the effect of coupling \nds containing few \sivs to antennas should include polarization measurements to experimentally quantify the impact of the emitter orientation. As it stands, no conclusive explanation of the recorded spectrum and the instant bleaching are available. Thus it is advised to repeat the whole procedure starting with a different \nd hosting few \sivs.

   % \section{Conclusions}

   Finally, we want to take a step back and discuss our results regarding \sivs in the context of the quest for reliable on-demand \sps. One of the properties that make \sivs interesting \spss for the purpose of calibrating detectors is their narrow \lw \zpl. Since detectors may have different sensitivity at different \wl, the actual \wl of the \zpl can become important. Thus, the existence of \hl containing emitters with \zpls of various \wls is interesting.
   In conjunction with the ability to preselect emitters according to their optical properties, candidate \nds tailored to a particular application can be found. An analogous argument is true for \vl with regards to the \cwl of emitters. However, at this point it needs to be emphasized that the preselection of emitters can quickly become an extremely tedious and time consuming process, hampering the feasibility of the approach. In particular when \nds are required that contain a single \siv, the odds of finding suitable emitters seem painfully small at the moment.

   In addition to questions of applications, our results regarding distribution of \siv properties raise further fundamental questions. While we managed to tie the observed variety of the emitters in \vl to varying strain in the \nds, a consistent explanation of the data in \hl is still missing. Here we hope that our results serve as input for further research into such questions.

   In our work we also explored the possibility of combining \nds containing \sivs with other nano structures in order to create an integrated \sps with favorable properties. To this end we utilize \pp methods to relocate selected \nds to target structures. Coupling \nds to \VCSELs is a promising approach to create a \sps whose activity can be controlled via the operation of the \VCSELs. In our work we demonstrated the first steps towards this goal. While we also identified the first roadblock preventing us from realizing a functioning integrated \sps for now, a promising approach to overcome it was suggested.

   While coupling to \VCSELs was aimed at increasing the control over the resulting integrated \sps, coupling to antennas aims at changing the properties of the emitters via the interaction between them and the antenna. Here we managed to demonstrate successful coupling between a \nd containing an ensemble of \sivs and an antenna. However, at this point we were not able to properly quantify the enhancement emitters experience through the coupling. However, the combination of numerical FTDT simulations and experimental methods has been established as a valuable method to predict the behavior of the coupled system. Thus we recommend this method to be incorporated in future attempts of coupling \sivs to antennas to create enhanced \sps.

   In conclusion, our research aims to contribute to the development of reliable on-demand \sps based on \sivs in two major ways. First, we establish properties of \sivs over large sample sets of \nds. In doing so, we add to the understanding of this type of \cc with the intention of subsequently improving our abilities to use \sivs as constituents of integrated \sps. To proceed in this direction we use individual \nds and couple their hosted \sivs to various nano structures. We demonstrate successful coupling and identify valuable suggestions for further research.
   We hope that our contributions add momentum to the development of \sps, to the development of novel calibration standards and photon detectors and ultimately, to the universal adoption of the quantum candela.
