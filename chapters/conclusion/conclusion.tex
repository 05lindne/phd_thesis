%!TEX root = ../../main.tex


 \chapter*{Summary and Conclusions}	\label{ch::conclusion}
 \addcontentsline{toc}{chapter}{Summary and Conclusions}
   % short summary of what happened in the thesis

   In this thesis we explore selected properties of the \si \cc (\siv) hosted in nano-sized diamond grains (\nds). Studying the \sivc in diamond is motivated by the fact that this particular \cc has surfaced as a promising candidate for realizing reliable on-demand \spss. Such non-classical light sources are key for the proper calibration of detectors and other instruments designed to operate reliably in the single-photon counting regime. This in turn is expected to drive a shift towards quantum based radiometric SI units, such as the quantum candela, which in the long run is expected to foster improvements in our ability to work with individual photons in a wide range of practical applications.

   The \siv has the potential to add considerable momentum to this cause. It is a stable narrow \lw emitter, emitting single photons with high intensity. Conveniently, \sivs operate at room temperature under ambient pressure and hence do not require extremely sophisticated experimental setups. When incorporated in nano-sized diamond individual \sivs can be identified and preselected according to their spectroscopic properties transfered using \pp techniques. 

   Our work can roughly be subdivided into two larger explorations. The first, reported in \Fref{ch::distribution}, revolves around charting the luminescence properties of a large set of emitters, allowing us to establish distributions of selected \siv properties. To our knowledge, our efforts result in the largest coherent examination of \sivs in \nds to date. In contrast to that the second exploration focuses on individual \nds hosting \sivs and our attempts to relocate them to different nano structures in order to achieve a coupling between \sivs and the host structure. \Fref{ch::coupling} details our efforts to realize a controllable integrated \sps by placing a \nd containing \sivs on top of a \vcsel (\VCSEL). Furthermore, we combine \nds with plasmonic nano-antennas in order to enhance the emission properties of the contained \sivs. A detailed summary of the results obtained in both explorations follows below.

   To enable our research, we synthesize \nds containing \sivs using a variety of different techniques. \Cvd (\CVD), \hpht (\HPHT) synthesis is used to fabricate \nds. In addition to the mentioned well established methods, wet-milling allows a high quantity of \nds to be produced in one process and therefore enables us to produce a sizable set of \nd samples. For this technique macroscopic diamond films are broken down to diamond grains of various sizes by vibrating steel beads in a vibrational mill. We improve the quality of \nd samples via annealing and oxidation. In particular Raman measurements show that contaminations due to non-diamond carbon can be significantly reduced.

   To investigate the obtained \nds, i.e.\ to study the optical properties of embedded \sivs we rely on optical excitation. In particular, confocal microscopy is used to collect emitted \fl. Samples are scanned in search for fluorescent emitters using \apds. Fluorescent light is further investigated with an attached spectrometer and single emitters are identified by means of a \HBT setup.


   % \section{\siv paper}

   With regards to the investigation of large sets of emitters we find a previously unknown strongly inhomogeneous distribution of \siv spectra in \nds. The distribution in particular suggests a partitioning of emitters according to the \wl and \lw of their \zpls (\ZPL) into two groups, \vl and \hl.

   \Hl consists of \ZPLs exhibiting a narrow \lw from below \SI{1}{nm} up to \SI{4}{nm} and a broad distribution of \cwls between \SIlist{710;840}{nm}.
   Compared to that, \vl is comprised of \ZPLs with a broad \lw between just below \SIlist{5; 17}{nm} and \cwls ranging from \SIrange{730}{742}{nm}.
   This data is consistent with previously measured SiV spectra \cite{Benedikter2017a,Neu2012}.
   For comparison, an \siv in unstrained bulk diamond exhibits a \lw of \SIrange{4}{5}{nm} at a \cwl of \SI{737.2}{nm}\cite{Arend2016a,Dietrich2014}.
   Based on ab initio density functional theory calculations we show that both the observed blue-shifts and red-shifts of the \ZPLs of \vl (as compared to an ideal, unstrained \siv) are consistently explained by strain in the diamond lattice.
   Further, we suggest, that \hl might be comprised of modified \sivs, the structure of which is currently unclear.

   The broad distribution of emission \wls found here covers all earlier results on spectroscopy of \sivs but considerably extends the range of known emission \wls. It further suggests that some single photon emitters in the range of \SIrange{715}{835}{\nm}, previously identified as Cr-, Ni- or Ni/Si-related, could in reality be strained or perturbed \sivs. Whereas single photon emission could be demonstrated for \sivc of both clusters, unveil further spectroscopic differences which set the two groups apart. For the \psb spectra in \vl we find one prominent peak at a shift of \SI{42}{meV}, which corresponds to a well-known feature assigned to non-localized lattice vibrations \cite{Larkins1971,Sternschulte1994}.
   In \hl we see accumulations of peaks, at around \SIlist{43;64;150;175}{meV}, which are consistent with sideband peaks reported in \cite{Sternschulte1994,Zaitsev2001,Neu2011}.

   % We further report photon autocorrelation measurements asserting the existence of single \sivs acting as \spss both in \hl and \vl.
   Investigating the time trace of \siv \pl, we find that some \sivs exhibit fluorescence intermittence, also known as blinking, with on-times ranging from several microseconds up to \SI{41}{s}. All but one single emitters in \hl exhibit blinking, where only one of the emitters in \vl exhibits blinking.

   Furthermore, we see an exponential distribution for bright-time intervals and a establish a log-normal distribution for dark-time intervals, consistent with research on single molecules \cite{Wong2013}.

   % \section{Coupling}

   After reporting the results of charting the properties of large sets of \nds and their contained \sivs, we move on to working with individual \nds.
   The ability to examine \sivs individually opens up the possibility to preselect emitters according to desired spectroscopic parameters, such as narrow \lws, high count-rates and single photon emission.

   Once such suitable candidates are identified, they can be transferred to different substrates with precision using \pp methods. Although these methods are difficult to execute, \sivs may be relocated and coupled to photonic structures where their extraordinary properties can be exploited to create \spss. In the scope of this thesis, \nds including suitable \sivs were identified and coupled to two different kinds of structures: \Vcsels (\VCSELs) and plasmonic nano-antennas.

   % \subsection{\Vcsels}

   Red AlGaInP-based oxide-confined \VCSELs are perfect candidates for the excitation of \sivs in a hybrid integrated \sps. Their physical size and the fact, that their output beam is perpendicular to the substrate implies that candidate \nds containing \sivs can simply be placed on top of the light-emitting region of the structure. Then, the \VCSELs output laser can be used to optically excite \sivs, steering emission of the \siv indirectly via the electrical operation of the \VCSEL. Through the use of suitable optical filters allowing only \siv \fl to emerge, a controlled \sps can be realized in principle.

   Although our initial attempt to realize a hybrid-integrated \sps by coupling \sivs to a \VCSEL is thwarted by large \sb emissions of the latter, a clear direction for further improvements was identified. In particular it seems promising to explore the idea of putting an additional gold layer on-top of the \VCSEL. If well-engineered it could be used as a mirror to suppress the \VCSEL \sb in the relevant \siv emission regime. Proper sideband suppression would open the door for an electrically driven, small scale single photon source. This system is interesting in the context of metrological applications, as it constitutes a promising building block for a portable device ready to calibrate single photon detectors \todo{add context here}.

   % \subsection{Antennas}

   With the experience gained by coupling individual \nds and their emitters to \VCSELs, we proceed to explore the possibilities of combining \sivs with plasmonic nano-antennas. Nano-antennas are very recent nano-scale devices designed to efficiently convert freely propagating optical radiation into localized energy and vice versa. Here we use gold double bowtie antennas on a gold substrate. This design allows the creation of a focused electromagnetic field in the gap in the center of the antenna, a so-called hot-spot. Placing emitters in such a hot-spot is known to increase their spontaneous emission rate provided their emission frequency is close to the resonance frequency of the antenna. This is known as Purcell effect and is used to improve the emission intensity of single emitters \cite{purcell1995spontaneous}.

   In our work we aim at placing a suitable \nd in the hot-spot of a double bowtie antenna in order to enhance the emission of the hosted \sivs.
   The term suitable is used to summarize both desirable spectroscopic properties such as narrow-bandwidth saturated single-photon emission as well as technical requirements such as \nd size and sufficient degree of isolation on the surface to enable \pp. 
   Feasible physical design parameters for the antenna meeting these requirements were determined using finite difference time domain (FDTD) simulations. Their resonance frequencies were calculated numerically and suitable antennas fabricated.
   We remark at this point, that a meaningful quantification of the Purcell emission enhancement induced by the coupling can in principle be achieved via intensity saturation measurements. By measuring the saturation intensities before and after insertion into the antenna and accounting for effects related to the polarization of emitters, the magnitude of the Purcell enhancement can be determined. Unfortunately, these methods are reserved for single emitters and do not apply for ensembles of \sivs. Naturally, the odds of identifying and addressing a \nd fulfilling all these criteria simultaneously are small. As a result identifying a perfect candidate for coupling is prohibitively time-consuming.

	 To mitigate this difficulty we decided to relax the condition of exactly one \siv per \nd and initiate our exploratory work with \nds containing several, potentially many active \sivs. Relying on \gtz measurements we identify two interesting classes of \nds. The first class consists of \nds containing large ensembles of \sivs acting as coherent emitters, which yield a flat response in the \gtz function. The second class of \nds we investigate features \nds hosting a small number of \sivs. As a result relevant \gtz measurements exhibit weak but discernible anti-bunching dips. Both classes have in common that relevant \nd specimen are significantly easier to obtain than \nds containing single \sivs. Thus \nds containing ensembles of \sivs as well as \nds containing few \sivs are both valid starting points for our exploration. To our knowledge, our experiments were the first attempts of coupling \sivs to plasmonic bowtie antennas.

   % \subsubsection{Ensembles}

   We begin by selecting a suitable \nd containing a large ensemble of \sivs and measure its spectrum in isolation. Our measurements of the \sivs hosted by a \nd coupled to the double bowtie antenna agree very well with predictions from FDTD simulations, asserting successful coupling facilitated by the \pp method. The remarkable agreement of the theoretical prediction and the experimentally recorded spectra for an ensemble of \sivs in a \nd make our simulation assisted approach very useful and a promising candidate for future explorations of antenna-emitter coupling.

   % \subsubsection{small number}

   Next we report our attempts to couple \nds hosting a limited number of \sivs to plasmonic gold double bowtie antennas. Suitable \nds show an anti-bunching dip in the \gtf in addition to count-rate saturation. This can be regarded an intermediate step towards using \nds containing single \sivs.

   Analogous to the process of coupling \nds with large ensembles of \sivs, we begin by recording the spectrum of the \nd in isolation. Next we relocate the \nd in question to the center of the antenna using the \pp method. Having assembled the integrated system, we record its spectrum in order to compare it with the simulation-assisted prediction focusing on dipole orientation effects. While some minor peaks of the measured spectrum can be associated with the simulation prediction, the major peaks seemed to have bleached.

While it is not possible to pinpoint exactly which circumstance caused the modification, several effects could influence the observed spectra. These include damage through electron radiation as well as additional fluorescing surface contaminations the \nd might have picked up during the \pp process. It also seems logical to assume that \nds containing only a few \sivs are less resilient to radiation damage than \nds containing large ensembles of \sivs.
   A property of \sivs which should not be neglected is their dipole orientation interacting with the antenna. Conducting dedicated FDTD simulations with a focus on different dipole orientations indicated a dramatic effect on the resulting spectra. Therefore, future experiments aiming to investigate the effect of coupling \nds containing single \sivs to antennas should include polarization measurements to experimentally quantify the impact of the emitter orientation. 

   % \section{Conclusions}

   Finally, we want to take a step back and discuss our results regarding \sivs in the context of the quest for reliable on-demand \spss. One of the properties that make \sivs interesting \spss for the purpose of calibrating detectors is their narrow \lw \zpl. Since detectors may have different sensitivity at different \wls, the actual \wl of the \zpl can become important. 
   The preselection of emitters is an extremely tedious and time consuming process, in particular when \nds are required that contain a single \siv. However, we showed that it is possible to preselect emitters according to their optical properties, and that candidate \nds tailored to a particular application can be found. In particular, the existence of \hl containing emitters with \zpls of various \wls is interesting. An analogous argument is true for \vl with regards to the \cwl of emitters.

   In addition to questions of applications, our results regarding distribution of \siv properties raise further fundamental questions. While a consistent explanation of the data in \hl is still missing, we managed to tie the observed variety of the emitters in \vl to varying strain in the \nds. Here we hope that our results serve as input for further research into such questions\todo{context here}.

   In our work we also explored the possibility of combining \nds containing \sivs with other nano structures in order to create a hybrid integrated \sps. To this end we utilize \pp methods to relocate selected \nds to target structures. Coupling \nds to \VCSELs is a promising approach to create a \sps which is controlled via the operation of the \VCSELs. In our work we demonstrated the first steps towards this goal. While we also identified the first roadblock preventing us from realizing a functioning hybrid integrated \sps for now, a promising approach to overcome the persisting challenges was suggested.

   While coupling \sivs to \VCSELs was aimed at increasing the control over the resulting \sps, coupling to antennas aims at changing the spectroscopic properties of the emitters via their interaction with the antenna. Here we managed to demonstrate successful coupling of a \nd containing an ensemble of \sivs and an antenna. While at this point we were not able to properly quantify the enhancement emitters experience through the coupling, the combination of numerical FDTD simulations and experimental methods has been established as a valuable method to predict the emission of the coupled system. Thus we recommend this method to be incorporated in future attempts of coupling \sivs to antennas to create enhanced \spss.

   In conclusion, our research aims to contribute to the development of reliable on-demand \spss based on \sivs in \nds in two major ways. First, we establish properties of \sivs over large sample sets of \nds. In doing so, we add to the understanding of this type of \cc with the intention of subsequently improving our abilities to use \sivs as constituents of integrated \spss. To proceed in this direction we use individual \nds and couple their hosted \sivs to various nano-structures. We demonstrate successful coupling and identify valuable suggestions for further research.
   Our contributions add momentum to the development of \spss, to the development of novel calibration standards and photon detectors and ultimately, to the universal adoption of the quantum candela.
