%!TEX root = ../../main.tex


\chapter{Introduction}	\label{ch::introduction}

	\section{contents}

		light are particles named photons
		single photons are useful
		applications include quantum information
		here we focus on quantum metrology
		quantum metrology applications

			quantum candela as redefinition of classical candela
			requires extremely well-calibrated detectors to make sense
			super calibration requires single photon sources

			ideal for calibration are narrow \lw emitters at various wavelengths
				in this thesis we investigate \sivs and find a distribution of wavelengths
				\sivs can be selected according to wavelenghts and deployed as calibration tools

				single photon sources should have sufficient Intensity


			normal light sources emit many photons at once
			quasi single photon sources have non-zero probability to emit more than one photon
			single photon sources are ideal for calibration

			heralded single photon sources give single photon sources with almost perfect certainty at the cost of lower intensities \cite{Bock2016}
			\sivs deliver high certainty and extreme intensities
			\sivs can be coupled to antennas to even further increase intensities
			thus \sivs are great for calibration purposes

			\sivs can be deployed as hybrid integrated single photon sources in conjuction with VCLS that are convenient to handle and operate
				they do not require sophisticated setups and operate at room temperature

			ideal for calibration are stable emitters that do not bleach
				\sivs in low strain bulk are stable
				\sivs in \nds can exhibit blinking, i.e.\ \fl intermittence
				however, the use of \nds has advantages
				 	increased collection efficiency
					high mobility, \nds can be moved by \pp
					high mobility enables special applications such as coupling with antennas

			in this thesis we:
				produces \sivs hosted in synthetic \nds
				investigate \sivs in \nds as single photon sources at room temperature
				explore luminescence properties with regards to potential applications in quantum metrology
				explore the possibilities of using \sivs with antennas


		\section{applications of single photon sources}

			metrology applications

			\begin{enumerate}
					\item reliable single photon sources can be used to calibrate detectors since their photon output is known exactly \cite{Vaigu2017}. Detectors have different sensitivities for different wavelengths. Narrow band emmitters such as \sivs allow to work with a specific wavelength, i.e.\ a well-defined and clean detector response \cite{Rodiek2017}. Narrow-band single photon sources with a variety of different \wl are ideal for calibration. This is possible, see distribution.
					\item redefinition and of classical SI candala based on single photon sources to achieve a high precision definition \cite{SIQUTE}.
			\end{enumerate}

			single photon sources emit single photons. Highly calibrated detectors can detect them faithfully and thus a definition of candela in terms of photon counts makes sense.


		\section{Introduction}

			The International System of Units (SI, abbreviated from the French Système international (d'unités)) emerged in the late $18^{th}$ century as a coherent metric system of measurement with rationally related units and simple rules for combining them \cite{zwinckels::1}. Since its inception it was improved and augmented continously as an effort to accomodate continuued scientific and technological progress. The current SI system is comprised of seven base units: The kilogram (\SI{}{\kg}), the second (\SI{}{\s}), the Ampere (\SI{}{\ampere}), the Kelvin (\SI{}{\kelvin}), the mole (\SI{}{\mole}) and the candela (\SI{}{\candela}). Currently a redefinition of four of base units (kilogram, mole, Kelvin, Ampere) in terms of fundamental constants is under way \cite{zwinckels::3, Milton, Martin (14 November 2016). Highlights in the work of the BIPM in 2016}. The proposed change will provide more universally realizable definitions of these base units, particularily for the measurement of electrical quantities \cite{zwinckels::paper}. It will also do away with the last base unit definition relying on a historic material artefact, the international prototype of the kilogram. As a result all base units will for the first time follow from one or more fundamental constants of nature.

			At the same time similar discussions regarding the SI base unit for luminous intensity, the candela, have emerged. It has been suggested that it can be improved by leveraging recent advances in classical radiometry and photometry as well as the development of novel quantum devices and techniques \cite{Cheung2007}.

			% zwinckels
			% The present definition of the candela links any photometric quantity to the corresponding radiometric quantity expressed in terms of the watt at one frequency only (corresponding to a wavelength of about 555nm in air) [6]. Radiometry describes the optical radiation and response in purely physical terms whereas photometry takes into account both the purely physical characteristics of the radiant power stimulating the visual system and the spectral responsivity of the latter. The net effect is intrinsically subjective and sets photometric quantities apart from purely physical quantities [6].

			Traditional applications relying on accurate photometric and radiometric measurements are light design, manufacturing and use of optical sources, detectors, optical components, couloured materials and optical radiation measuring equipment \cite{zwinckels::paper}. In the classical regime of optical radiation high flux levels dominate. Here primary optical radiation scales for sources and detectors are generally based on cryogenic radiometry establishing a link to the SI units of electricity \cite{Cheung2007::5}. Other calculabe sources such as synchrotrons and blackbody radiators can serve as primary source scales in the ultraviolet and deep-UV regions by establishing tracability to SI units of thermometry, electricty and length \cite{zwinckels::paper, Cheung2007}.

			Scaling down to the quantum world of radiometry is associated with a loss of measurment accuracy. In this regime dedicated photon counting techniques are required to deal with the challenge of low flux levels. Since they rely on counting photons directly, the can provide efficient and traceable measurements and improved uncertainties. For high-accuracy absolute radiometry in this regime predictable single or quasi-single-photon sources and photon detectors are called for.

			% zwinckels



			% The traditional applications requiring high-accuracy photometric and radiometric measurements are lighting design, manufacturing and use of optical sources, detectors, optical components, coloured materials and optical radiation measuring equipment. In this classical world, the primary optical radiation scales are generally based on cryogenic radiometry with traceability linked to the SI units of electricity. For work in the ultraviolet (UV), deep-UV and infrared regions, primary source scales are also based on calculable sources such as synchrotron and Planckian radiators with traceability to the SI units of thermometry, electricity and length. The standard uncertainties in the primary detector scale are currently around the 0.005 \% level, and are based purely on characterization and calculation. The primary source scale, particularly in the visible range, is derived from the primary detector scale through the use of filter radiometry, and uncertainties are currently around the 0.2 \% level. The detector and source scales are established at discrete wavelengths and then made spectrally continuous through the characterization of intermediate standards, such as trap detectors and blackbody sources.

			% In scaling down to the photon counting regime there is an unavoidable loss in accuracy. For high-accuracy absolute radiometry at the quantum level, predictable or quasi-single- photon sources and photon detectors are needed. Quantum optics techniques offer improved uncertainties in this regime, as they are directly applicable to measurements at photon counting levels and can provide a direct, and therefore more effective, way of delivering measurements in the photon counting regime.

			The recent advances being made in managing and counting individual photons and producing single-photon sources show tremendous promise of producing within a few years a radiant flux with a well-established number of photons per second with an unprecedented precision and accuracy beyond the standard quantum limit [7,8]. Moreover, the ability to reliably manipulate individual photons will foster the development of new types of instruments that will require, in turn, advances in metrology to create new quantum-based calibration methods and standards. For these reasons, a reformulation of the candela has been proposed in terms of photon units. This reformulation is considered to be a small, but useful step in the future direction of photometry, radiometry and the ‘candela’ in the quantum world. For instance, the emerging fields of nanotechnology and quantum communication are promising new technologies. While their immediate challenges are for new metrological approaches for reliable characterization of properties at the nanoscale, it is foreseen that to advance the progress of these technologies in building verifiable large-scale systems, they will need accurate measurements traceable to the SI using quantum- based radiometric units.

			% Responding to these challenging needs for traceable, accurate measurements at the level of single or few photons is having an impact on the focus of metrology programmes of he national metrology institutes (NMIs). This technological revolution may lead to new quantum-based realizations of the SI units with improved accuracy. The inclusion of photon aspects in the explanations of the definition of the candela might be a critical enabler for these emerging technologies. However, there is no unanimity within the CCPR on this point. In any event, the CCPR has been proactive by discussing here a possible reformulation of the candela to meet the metrological needs of these new quantum-based technologies.

			% The major problem of modern radiometry is how to cover the wide dynamic range of radiometric measurements with reliable and traceable methods. As we have seen, for photometry, this dynamic range extends over more than 15 decades. The problem of dynamic range in photometry is ‘solved’ by the eye itself, which has different types of receptors operating at different luminance levels. Thus, for photometric measurement and for photometric units, the links to the SI are realizable by the present CIPM definition of the candela and the CIE photometric system. The situation is different for radiometry because different types of instrumentation based on different physical principles are used to measure radiometric quantities at different levels of flux. Thus, for radiometric measurement and for radiometric units, the links to the SI are more difficult to establish over the full dynamic range.
			%
			% In the classical world of radiometry at high flux levels, the primary optical radiation scales for sources and detectors are generally based on cryogenic radiometry with traceability linked to the SI units of electricity. In the visible range, the detector and source scales are established at discrete wavelengths in the 0.1 mW to 1 mW regime with state-of-the- art uncertainties around the 0.005 \% level. In scaling down to the quantum world of radiometry at very low flux levels, photon counting techniques are used, with an unavoidable degradation in accuracy. For high-accuracy absolute radiometry at the quantum level, predictable or quasi-single-photon sources and photon detectors are needed. Quantum optics techniques offer improved uncertainties in this photon counting regime, as they are directly applicable to measurements at photon counting levels and can provide a direct, and therefore more effective, way of delivering traceable measurements in this challenging regime.

			A key requirement for the progress of quantum information technology is the development of sources that deterministically produce single photons upon request (on-demand source). Recently, single-photon sources and entangled-photon sources have become available, where the key issue is the non-Poisson generation of single photons.
			A laser beam can be described by a single-mode coherent state with Poissonian photon-number distribution, p(n) = (μn /n!)e−μ , where μ is a mean photon number in the beam. Thus, a highly attenuated laser pulse with very small μ approximates quite well a single-photon Fock state with the probability ratio of multiple photons to a single photon going to 0 as μ → 0. Unfortunately, the fraction of vacuum states then increases dramatically. Moreover, the mean photon number cannot be made arbitrarily low because of detector dark counts.

			Quasi-single-photon states can be prepared more efficiently by using signal and idler photon pairs generated by spontaneous parametric down-conversion (SPDC) [111]. SPDC is a deterministic single-photon source (see also section 5.2), where the number of photons in one mode is thermally distributed and the total number in all modes is Poissonian distributed. The key feature is a strong time correlation between photons in the pair. Ideally, if a photon counter detects one photon in the idler path then, for an extremely short time interval, of the order of hundreds of femtoseconds, the other photon in the pair is in the signal path. However, losses in the signal beam and dark counts of the trigger detector can result in no photon in the signal beam even if the trigger detector has clicked. SPDC photons can be satisfactorily used as a heralded single-photon source, albeit random, and provide a useful approximation capable of demonstrating single photon-ness. In general, the probability of having multi-photon states is rather low, mainly because the efficiency of the overall SPDC conversion is very poor. Eventually, these states can be effectively eliminated by using techniques available in the literature [112, 113]. In conclusion, a SPDC quasi-single-photon source is characterized by a substantial reduction in the portion of vacuum contributions, i.e. empty signals, compared with an attenuated laser.
			Colour centres in synthetic diamond with a substitutional nitrogen atom and a vacancy at any adjacent lattice position represent an interesting single-photon source with strong anti-bunching and a spectral width about 1nm at room temperature [114].

			In quantum dots, i.e. semiconductor nanostructures characterized by a two- or more-level electronic system, the photon is emitted by recombination of an electron–hole pair that can be created by optical pumping or by an electric current [115]. The chosen material determines the wavelength of the emitted beam while the spectral width is a function of the number of excited energy levels and the average number of created electron–hole pairs.
			Single-photon-like states can also be generated by radiative transitions between electronic levels of a single atom (ion) or molecule caught in a trap and placed inside (or sent into) an optical cavity, interacting both with the excitation laser beam and the vacuum field of the cavity [116]. These single- photon sources have desirable properties such as a narrow spectrum and high collection efficiency due to the presence of the cavity. However, the practical feasibility of such sources is still low because of their technological complexity (among others, high vacuum is needed). Furthermore, semiconductor quantum dots and colour centres face the problems of spectral dephasing and inhomogeneity, which make it difficult to find independent emitters that generate indistinguishable photons for applications, such as quantum computing.

			Organic molecules in a crystalline host matrix offer another candidate as a practical scalable single-photon source [117]. At low temperatures, some molecular transitions become lifetime limited and offer almost unity quantum yield. Recently, two independent Fourier-limited solid-state single- photon sources have been demonstrated [118]. The solid-state arrangement of this approach enables very long measurement times using the same emitter, practical frequency tunability of individual molecules, and a straightforward method for scaling and miniaturization.

			As a consequence of the number of technologies being pursued to develop single-photon emitters with different source properties, a number of new metrics are needed to assess these sources. Suitable metrics include the methods and measurement facilities that have been developed for quantitative characterization of statistical properties of light and non-classical signature of single- photon sources, such as their anti-bunching behaviour, purity, degree of indistinguishability and ability to be used for entanglement [113, 119]. Capabilities for quantifying the classical and quantum characteristics of photons have also been demonstrated in the measurement of single photon/correlated photons with Hanbury-Brown–Twiss, two- photon and Michelson interferometers [120, 121].
			The present challenge is to improve the accuracy of these measurements and enable them to characterize sources under varying conditions of pumping, temperature, geometry and wavelength. As an example, a good measure of the quality of a single-photon source is its second-order autocorrelation function, i.e. the correlation measured in a Hanbury-Brown– Twiss-type experiment. The signature of true (ideal) single- photon emission is an absence of any coincidences at zero time difference between the two detectors.

			The evolution of photometry, radiometry and the candela has been reviewed in terms of needs and developments in both the classical and the quantum world. The metrological basis of these physical quantities and associated SI base unit is linked to the important biological process of human vision, which is generally described in terms of photon interactions. For several decades, the advances in this field of metrology have been spurred by research and development of new and improved source-based and detector-based primary methods. The current state-of-the-art uncertainties of these measurements largely meet the diverse community of users in the classical world of lighting, manufacturing, commerce and health and safety applications. However, there have been recent developments in producing novel single-photon sources and single-photon detectors. The measurement challenges in characterizing these new ‘photon on demand’ sources and PNR detectors have been identified. The growth and potential of new quantum-based tools that exploit these new sources and detectors are largely limited by the lack of traceablequantumopticalmetrologytoolsandtechniques. The definition and practical realization of the candela and the other photometric and radiometric units are regularly reviewed by the Consultative Committee for Photometry and Radiometry (CCPR) to ensure that they reflect current best measurement practices and meet the existing and emerging needs of its user community. The possibility of a reformulation of the candela has been discussed here in terms of expanding its scope of application from meeting existing and future needs in classical photometry and radiometry to including future needs in quantum radiometry. This would build on the existing official definition in terms of radiant intensity by providing an explanatory note in terms of fundamental photon units. This reformulation is considered by members of the photon counting community to be a small, but useful step in the future direction of photometry, radiometry and the ‘candela’ in the quantum world. In addition to addressing the metrological needs of emerging quantum-based optical technologies and applications, this reformulation would also be in harmony with the proposed redefinitions of four of the seven SI base units—the kilogram, ampere, kelvin and mole—in terms of fundamental constants in a quantum-based SI system.
			%
			% % rodiek2017
			% Single-photon sources have the potential to be used in a wide field of applications [1,2]. Well known and widely discussed is their use in quantum key distribution, quantum computing, and quantum-enhanced optical measurements [3].
			% In this paper, we deal with another important application of single-photon sources, i.e., their use in radiometry [4]. In principle, single- photon sources have the potential to become a new type of standard photon source [5] as there are, in the classical regime, the blackbody radiator and the synchrotron radiation source.
			% The output power phi of an ideal single-photon source, which emits only one photon per excitation pulse, is simply given by the formula
			% phi f h c / lambda, where f is the repetition rate of the excitation laser, h is the Planck constant, c is the speed of light, and lambda is the wavelength of the emitted radiation.
			%
			% However, the conditions for such a standard source are difficult to realize in practice, because a source with a quantum efficiency of 100 \%, a perfect purity of the single-photon emission, i.e., g2, and a collection effi- ciency of 100 \% is required. In any case, single-photon sources are ideal sources for the calibration of single-photon detectors, because the influence of photon statistics on the calibration results is omitted [6,7]. In this paper, we present a single-photon source absolutely calibrated by a classical standard detector and a cali- brated spectroradiometer and hence traceable to a national stan- dard for optical radiant flux via an unbroken traceability chain. This is considered to be the first step toward the realization of a deterministic absolute single-photon source. Furthermore, the model for the calculation of the absolute spectral photon flux from experimental data is presented in detail.
			%
			% % vaigu2017
			% Quantum information technology has been the major driving force for the development of single emitter single-photon sources (SPSs) [1, 2] with prominent applications in quantum key dis- tribution, all optical quantum computation and quantum simula- tions. On-demand SPS can also allow for novel applications in other research fields such as quantum imaging and metrology [3].
			%
			% In metrology, a regulated stream of single photons can provide an accurate and predictable scale for low optical power levels. The work on on-demand SPSs can contribute to the realization of photon number based measurement applications of photobio- logical and photochemical quantities and their related units [4]. Optical quantum technologies require careful characterization of photodetectors by traceable and reliable measure- ments at the few-photon level. The idea of traceability means that the measurement results can be linked to the international system of units (SI), typically by the means of a standard, such that different measurements of the same quantity can always be reliably compared with each other. Currently, the realized optical power level at visible wavelengths lies in the range of 100 μW to 1 mW [5], and can be extrapolated downwards by a few orders of magnitude by relying on the linear respon- sivity of silicon detectors. Whilst this is sufficient for tradi- tional needs, these obtained power levels are not sufficient for single-photon applications.
			%
			% A vast selection of SPSs, including attenuated lasers [6–8], parametric down-conversion based heralded single-photon sources [9–12], and sources based on a variety of single photon emitters such as colour centres in diamond, organic molecules, and quantum dots [13–20], have been successfully demonstrated. Much progress has been made by improving the characteristics of these SPSs in terms of purity, indistin- guishability, and collection efficiency [21–25]. An on-demand SPS, which emits exactly one photon upon a trigger, is yet to be realized. Existing implementations of such a device still face a number of technical challenges, including the emitter quantum efficiency (excitation to photon emission ratio), the collection efficiency and other losses in the optical system.
			%
			% This paper describes a method that allows us to realize a SI traceable SPS with current technology. Our method takes advantage of a very sensitive conventional gauge photode- tector [26, 27]. A measurement of the SPS’s absolute optical power with this detector eliminates the need for a precise knowledge of the SPS characteristics. By changing the pump laser repetition rate, the photon flux of the SPS can be tuned in a controlled way. This gives us a direct way of linking con- ventional optical power levels obtainable with specifically designed analog-mode detectors down to low photon flux levels needed for single-photon detectors.

		\section{psu}

			Kontoinhaber: 	Effektboutique
			Bankname: 	Volksbank Esslingen
			IBAN: 	DE72611901100244357005
			BIC: 	GENODES1ESS
			229,00 €
			Ihre Bestellnummer lautet: BE12813.
