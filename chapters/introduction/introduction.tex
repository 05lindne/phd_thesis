%!TEX root = ../../main.tex


\chapter{Introduction}	\label{ch::introduction}
	
Within the last few years the topic of quantum technologies has witnessed a large progress in fields like quantum cryptography, quantum computation and quantum metrology.
Requirements for these technologies are ideally single photons which are produced \enquote{on demand} i.e. deterministically and have well defined properties.
For instance, quantum computing places restraints on the photon sources, called the diVincenzo criteria (\todo[fancyline]{enter diVincenzo criteria}).
The progress of these technologies calls for measurement standards to make the individual measurements comparable???
\\
The candela is the SI (syst\`eme internationale) unit for optical radiation \cite{Cheung2007} and is the only unit which is linked to physiological processes, namely the varying sensitivity of the human eye to radiation of different frequencies.
It is a photometric quantity, meaning that a physical measurement of the light in terms of luminous intensity represents the visual sensation experienced by a human observer exposed to the same source of light.
It is one of the base units since the system was first introduced.
In the latest definition, the candela is linked to the unit watt.
The current definition of the candela (cd) is the following:
\enquote{The candela is the luminous intensity, in a given direction, of a source that emits monochromatic radiation of frequency \num{540e12} hertz and that has a radiant intensity in that direction of 1/683 watt per steradian.}\cite{NistSIunits}
Advances in the quantum technologies which operate in the photon-counting regime would profit from a redefinition of the candela in terms of photon number, named \enquote{quantum candela}.
\\
The term \enquote{quantum candela} is used to describe a reformulation of the candela by defining it via a countable number of photons.
This new definition would be of a form that converts the current definition into
\begin{equation}
	P=nh\nu
\end{equation}

where 

\begin{align*}
	P&=\text{power}=\text{\SI{1/683}{\watt} exactly}\\
	n&=\text{number of photons per second}\\
	h&=\text{Planck's constant}=\text{\SI{6.6260693+-11e-34}{\joule\second}}\\
	\nu&=\text{photon frequency}=\text{\SI{540e12}{\hertz} exactly}
\end{align*}

which yields

\[
n=\text{\SI{4.0919429+-7e15}{\per\second}}
\]
\cite{Cheung2007}

The number of photons of all wavelengths emitted or contained in a given beam of light is given by:
\begin{equation}
	N_p=\int\frac{E_{\nu}}{h\nu}\diff\nu =\int\frac{\lambda E_{\lambda} }{hc} \diff\lambda
\end{equation}






