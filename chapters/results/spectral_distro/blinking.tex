%!TEX root = ../../../main.tex

	\section{Photostability} \label{subsec::photostab}

		As mentioned in the previous section, the single photon count rates observed from the investigated \sivs varies strongly between a few thousand to a few \SI{100000}{cps}.
		To further investigate the count rate, the luminescence time trajectory of the emitters which exhibit a dip at \gtz is evaluated.
		It is found that some of the observed emitters exhibit fluorescence intermittence, also called blinking. \Fref{fig::blink} illustrate the effect.
		Blinking is attributed to temporal ionization of the color center during optical excitation, forming an optically inactive charge state \cite{Jantzen2016,Neu2012a,Gali2013}.
		Therefore the emitters change between states of higher and lower emission, i.e.\ brighter and darker states, called blinking levels.

		\begin{figure}[!htb]
			\begin{subfigure}[tp]{ 0.49\linewidth}
				\centering
				\testbox{\includegraphics[trim = 0 0 0 0,  clip= true, width = \pairplotwide]{./pics/g2zuSpektrum8_15_5_countrate_0_01_conv_screenshot.png}}
				\caption{}\label{subfig::blink_long}
			\end{subfigure}
			\hfill
			\begin{subfigure}[tp]{ 0.49\linewidth}
				\centering
				\testbox{\includegraphics[trim = 0 0 0 0,  clip= true, width =\pairplotwide]{./pics/g2zuSpektrum8_15_5_countrate_0_01_conv_detail.pdf}}
				\caption{}\label{subfig::blink_short}
			\end{subfigure}
			\caption[Time trace of a single emitter.]{(a) Time trace of the single emitter H1, which exhibits strong blinking. The variation of the count rate in the upper state is attributed to a drift of the setup. (b) Detail of the time trace of the same emitter.}
			\label{fig::blink}
		\end{figure}

		The photon count time trace of emitter \emnarrow is shown in \Fref{fig::blink}.
		In the overview picture (\Fref{subfig::blink_long}), a few blinking dips can be seen with time intervals of up to a few minutes.
		The fact that the count rate never drops down to the dark count rate lets us assume, that there are at least two \sivs present, one exhibiting fluorescence intermittence and one exhibiting a stable emission.
		When zooming in, shorter time intervals are observable (\Fref{subfig::blink_short}).
		The time intervals range from a few tens of milliseconds up to a few seconds with a few outliers exhibiting very long time intervals up to a few hundred seconds.
		
		The bright and dark times exhibit different probability distribution functions and with that, different characteristic time constants.
		In \Fref{fig::fit_blink_distr} the time intervals of the emitter are shown as small vertical dashed red lines and solid blue lines for the bright and dark state respectively.

		\begin{figure}[!htb]
			\centering
			\testbox{\includegraphics[trim = 0 0 0 0,  clip= true, width = \oneimagewide]{./pics/retention_no_histo_xlim2_5.pdf}}
			\caption[Distributions of bright and dark state intervals]{Time intervals of the single emitter exhibiting the highest blinking rate in the bright (red) and dark (blue) states. Each flank of the blinking state was individually read out from the \pl time trace. On the horizontal axis small vertical lines represent the individual data points of the bright/dark time intervals. The blue and red dashed curves represent kernel density estimations of the distribution of time intervals of the dark and bright states, respectively. The y-axis is scaled to the normalized kernel density estimate. The red solid line is an exponential fit of the bright state time intervals whereas the blue solid line is a log-normal fit of the dark state time intervals. These fitting functions were chosen because they provide the best agreement with the data using a Kolomogorov-Smirnov test with respect to other functions (p-values: bright state (red) 0.92, dark state (blue) 0.77).}
			\label{fig::fit_blink_distr}
		\end{figure}

		Outliers with very long time intervals are ommited here.
		The dashed lines are kernel density estimators of the distribution of the respective time intervals.
		This implies that every data point is represented with a Gaussian function and the resulting functions are added up to model the whole data.
		The red solid line is an exponential fit of the distribution of time intervals in the bright state.
		The high p-value of \num{0.92} confirms the goodness of the fit.
		The median time interval in the bright state obtained by the exponential fit amounts to \SI[separate-uncertainty]{0.09}{s}.
		While other literature on solid state quantum emitters reports an exponential probability distribution for both time intervals in bright and dark states\cite{Bradac2010,Berhane2017}, we found a log-normal probability distribution for the time interval in the dark state.
		The solid blue line in \Fref{fig::fit_blink_distr} is a log-normal fit of the distribution of the time intervals in the dark state.
		A Kolomogorov-Smirnov test yields a p-value of \num{0.77} for the log-normal fit and is by far the best model to describe the data distribution.
		For comparison: The p-value of an exponential fit amounts to \num{0.36}.
		The median time interval in the dark state obtained by the log-normal fit is determined as \SI{0.10}{s}, therefore being close to the median time interval in the bright state.
		Very long time intervals are not shown in the plot for better visualization of the small timescales, however these long time intervals are included in the fit.
		The longest measured time interval amounted to \SI{41.14}{s} and occurred in the dark state.
		
		Measurements of blinking time intervals in \cite{Jantzen2016} and \cite{Neu2012a} report time intervals between about \SIrange{0.03}{1}{s}, and \SI{0.1}{s} to \SI{2}{min}, respectively.
		These findings are in good agreement with our measurements.
		
		In general, excitation and emission process is mediated by charge separation (i.e.\ excitons).
		If an electron or hole is localized far enough that there is sufficiently negligible overlap with the wave function of the remaining carrier, blinking occurs \cite{Efros2016}.
		We explain the observed blinking as a manifestation of the local crystal disorder due to dislocations and impurities which act as a trap for the excited electron and therefore switch the emitter to the dark state.
		The capture of the electron of the exciton by traps due to local crystal disorders may inhibit recombination and therefore induce the dark state \cite{Bradac2010}.
		The assumption that dislocations and impurities are responsible for blinking emitters is in agreement with our findings reported in \ref{ch::crystal_quality}.
		
		Research of blinking rhodamine molecules confirms power law distributed bright state times and log-normal distributed dark state times \cite{Wong2013}.
		The log-normal distribution is explained by a Gaussian distribution of activation barriers of the electron transfer to trap states in the surrounding material \cite{Albery1985}.
		It hints towards a recapture of the electron via multi-phonon relaxation channels.
