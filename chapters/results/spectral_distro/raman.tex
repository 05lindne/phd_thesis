%!TEX root = ../../main.tex

% \section{Diamond Characteristics}

% 	\subsection{Raman Measurements} \label{subsec::raman}

% 	For the Raman measurements the same layout of the setup described in \Fref{sec::confocal} is used.
% 	As excitation lightsource, a \SI{532}{nm} diode laser is used (IO \todo[fancyline]{type}).
% 	It provides single (frequency) mode laser light, which is a prerequisite for Raman investigations.
% 	The beamsplitter is a dichroic mirror (DRLP645\todo[fancyline]{type}), the laser light is additionally filtered out with a 532 Notch filter\todo[fancyline]{type} in the detection path in front of the single mode fiber instead of the otherwise used longpass filter.
% 	With these adaptions, the combination of the confocal unit and the spectrometer serve as a Raman spectrometer.
% 	As the diamond Raman line is very narrow (~xx\todo[fancyline]{check number}), the \SI[per-mode=symbol]{600}{\lines\per\mm} grating is used for a first overview; for more detailed measurements the \SI[per-mode=symbol]{1200}{\lines\per\mm} and \SI[per-mode=symbol]{1800}{\lines\per\mm} gratings are used.
