%!TEX root = ../../../main.tex

\chapter[Spectral Distribution]{Spectral Distribution of \sivs in Nanodiamonds}	\label{ch::distribution}


	In the following chapter, spectroscopic measurements of the \sivs are described.
	% Unless explicitly otherwise stated, the results in this paper report measurements of the milled nanodiamonds containing \textit{in-situ} incorporated \sivs.
	At first, a short introduction will be given.
	The additional theory, which goes beyond the scope of the explained theory in \autoref{ch::theory} and additional experimental equipment will be explained.

	Throughout the work on this thesis, many different \nd samples were investigated.
	While the ideal \siv in unstrained bulk diamond has a \lw of about \SIrange{4}{5}{nm} at a \cwl of \SI{738}{nm} at room temperature, it was observed that the \cwl of the \siv in \nd varies \cite{}\todo{cite elke}.
	At some point, it was thought that chromium centers are responsible for some of those shifted lines.
	With all the work performed on various \nds we can now say, that all the \ZPLs we see are silicon related.
	It is possible that during \textit{in-situ} growth atoms other than \si are incorporated into the diamond lattice. 
	In implanted high-purity diamond the possibility that the diamond contains color centers other than the implanted ones is very narrow.
	Therefore, we used implanted samples to confirm our findings we obtained with the in-situ implanted ones.



