%!TEX root = ../../main.tex

\chapter{Silicon Vacancy Centers in Diamond}	\label{ch::sivs}
\chaptermark{SiV Centers}

\begin{remark}
    \begin{itemize}
      \item Fill in used energies for \siv excitation.
    \end{itemize}
\end{remark}

\section{Diamond as a host lattice}

  Diamond is a metastable modification of carbon which is, in fact, stable under normal conditions \cite{steinmetz::52}. Carbon atoms form strong $sp3$-bonds with each other in a tetraedal arrangement of neighboring atoms. The resulting $sp3$-hybridized lattice is of exceptional mechanical stability, making diamond the hardest known material \cite{?}. The crystal structure can also be interpreted as a face-centered cubic (fcc) lattice with two carbon atoms in the primitive Bravais cell, situated at $(0,0,0) \ a$ and $ (\frac{1}{4}, \frac{1}{4}, \frac{1}{4}) \ a$ with $a = \SI{3.567}{\angstrom}$ denoting the lattice constant \cite{steinmetz::56}. \autoref{fig::diamond_lattice} illustrates the structure.

  \begin{figure}[htbp]
		\centering
		\testbox{\includegraphics[trim = 0 0 0 0,  clip= true, width = 0.3\textwidth]{./pics/diamond_lattice.png}}
		\caption[Face-centered cubic diamond lattice]{Face-centered cubic diamond lattice. Note the tetraedal arrangements of carbon atoms. Figure reproduced with permission from \cite{janine::thesis}.}
		\label{fig::diamond_lattice}
	\end{figure}

  The valence and conduction bands of diamond are separated energetically by a large direct band gap of \SI{7.3}{\eV} while its indirect band gap amounts to \SI{5.5}{\eV} \cite{steinmetz::57, neu::91}. As a result diamonds are transparent for light of all wavelengths larger than \SI{230}{\nm} \cite{neu::87}. This transparent quality makes diamond an ideal host material for various optically active lattice defects or impurities. These induce a wide range of discrete energy levels accomodated by the sizable band gap. The absorption of optically active impurities or impurity complexes gives rise to the color of diamonds, thus these impurities are commonly termed color centers \cite{neu::thesis}. Due to the exceptional mechanical stability of diamond color centers too are very stable, another important property enabling optical applications.

  A property of diamond, detremental to some optical applications, is its large refractive index with values of $2.49$ at \SI{360}{\nm} and $2.4$ at \SI{800}{\nm} respectively \cite{steinmetz::58}. Thus, a portion of the flourescent light escaping from the diamond is reflected back into it, effectively reducing the efficiency of light extraction. If \nds smaller than the wavelenght of the light to be collected are used, internal reflection is supressed and the extraction efficiency can be increased \cite{neu::49}.

  \begin{figure}[htbp]
		\centering
		\testbox{\includegraphics[trim = 0 0 0 0, clip= true, width = 0.3\textwidth]{./pics/diamond_refraction.png}}
		\caption[Reduced light extraction efficiency of diamond due to refraction]{Light from a flourescent emmitter inside the diamond (green dot) undergoes reflection at the diamond-air interface. Figure reproduced with permission from \cite{neu::thesis}.}
		\label{fig::diamond_refraction}
	\end{figure}

\section{Classification of diamond}

  Two major approches for classifying diamond are commonly encountered. First, classification according to the presence or absence of certain impurities or impurity complexes. Second, classification based on different diamond crystallinities observed. In the following both classification systems are briefly introduced.

  \subsection{Classification by impurities}

    Impurities or complexes of impurities in the diamond lattice can be optically active and thus change the optical properties of diamond. Most strikingly perhaps is the appearence of color in otherwise colorless diamond due to a sufficient concentration of such defects. Using IR absorption spectroscopy the degree of nitrogen impurtities can be determined. It is used to subdivide diamonds into distinct groups named Type I and Type II \cite{janine::148, janine::149}. The groups are further subdivided as follows:

    \begin{itemize}
          \item Type Ia: With a nitrogen concentration of up to \SI{3000}{ppm} most natural occuring diamonds belong to this group \cite{steinmetz::58}. Nitrogen appears arranged predominantly in aggregate clusters forming complexes of impurties. These complexes are optically active, absorbing light in the blue range of the visible spectrum. Consequently Type Ia diamonds often exhibit a yellow to brownish coloration.

          \item Type Ib: With concentrations of up to \SI{500}{ppm} nitrogen atoms appear predominantly in isolation, replacing individual carbon atoms in the diamond lattice. In addition to absorbing visible blue light, green is being absorbed as well. Type Ib diamond thus exibits intensified the yellow or brownish coloration. While only \SI{0.1}{\percent} of naturally occuring diamond fall into this class, almost all synthetic diamonds created using the \hpht (\HPHT) method are of Type Ib \cite{steinmetz::58}.
    \end{itemize}

    While Type I diamond exhibits an appreciable concentration of nitrogen, Type II diamonds lack nitrogen entirely. Type II diamond is divided into two subgroups according to the presence or absence of boron as follows:

    \begin{itemize}
      \item Type IIa: Can be considered pure as they lack boron impurities and other optically active defects \cite{neu::84}. They thus are colorless. Up to \SI{2}{\percent} of natrually occuring diamond and most diamonds synthetically created using the \cvd (\CVD) method are of Type IIa \cite{steinmetz::58}.
      \item Type IIb: Contains appreciable concentrations of boron atoms replacing individual carbon atoms in the diamond lattice. Boron defects are optically active absorbing visible light ranging from red to yellow. Depending on the Boron concentration blue to grey colorations are observed. Furthermore, diamond is turned from an insulator to a efficient p-type semiconductor in the presence of boron impurities \cite{steinmetz::59}.
    \end{itemize}

    We remark that for many modern applications of diamonds the presented ``classic" categorisation of diamond is not enough. In these cases a precise quantification of the concentration and nature of various relevant impurities is called for \cite{neu::85, neu::86}.

    In this section we also briefly touched upon the \CVD and \HPHT methods, two approaches to synthetically produce diamonds. Both are relevant for this thesis and are explained in detail in \autoref{ch::fabrication_nanodiamonds}.

  \subsection{Classification by crystallinity}

    Up unitl now, the discussion assumed that diamond forms a lattice consisting of one giant single crystal. However, other crystallinities are possible and can be used to classify diamond.
    They range from mono or single crystals to polycrystalline, nanocrystalline or even ultra-nanocrystalline diamond films \cite{steinmetz::62}. This classification is particularirly useful for synthesized diamon as will be discussed in \autoref{ch::fabrication_nanodiamonds}. \autoref{tab::diamond_grain_sizes} summarizes the different sizes of diamonds or diamond films which can be achieved using variations of the \CVD method.

    Diamond films consist of isolated diamond grain of random orientation with sp2 hybridized grain boundaries and graphit-like inclusions \cite{janine::thesis}. Carbon present in non-diamond phases, e.g.\ graphite or amorphous carbon gives rise to detremental light absorption while crystal boundaries lead to increased scattering losses. As the size of diamond crystallites get smaller, the ratio of non-diamond carbon to diamond carbon increases. Thus losses are most pronounced for the smallest grain diamond films.

    \begin{table}[htbp]
  		\centering
  		\caption[Classification of diamond synthesized by \cvd]{Classification of diamonds sythensized using \CVD \cite{steinmetz::thesis}.} \label{tab::diamond_grain_sizes}
  			\begin{tabular}{lc}
  			\toprule
  			Crystallity & Grain size  \\
  			\midrule
  			monocrystalline & arbitrary \\
  			polycrystalline & \SI{50}{\nm} to \SI{10}{\micro\meter} \\
  			nanocrystalline & \SI{10}{\nm} to \SI{50}{\nm} \\
  			ultra-nanocrystalline & $<$ \SI{10}{\nm} \\
  			\bottomrule
  			\end{tabular}
  	\end{table}

\section{\sivs} \label{sec::siv}

  A \cc is an optically active point-defect in a crystall lattice, capable of absorbing and emitting light. Defects can consist of one or several vacant lattice sites, foreign atoms replacing lattice atoms or a combination thereof. If the precence of a defect induces discrete energy levels located in the band gap of the host material, the \cc can be interpreted as its own quantum system. In other words, the \cc can be viewed as a single isolated and localized artificial atom embedded in a host matrix. As such it is able to absorb light and emit single photons by means of flouresence.

  Compared to alternative single photon sources like single atoms \cite{steinmetz::27}, Ions \cite{steinmetz::29} or individual quantum dots \cite{steinmetz::32, steinmetz::34}, \ccs offer a couple of advantages due to their solid state environment. As a result of the high mechanical stability of the host lattice \ccs exhibit increased photo-stability, in particlar compared to organic molecules as light sources. Furthermore the host lattices offers protection for \ccs from detremental interactions with agressive free molecules \cite{steimetz::36}. Lastly, \ccs can be handled and investigated at room temperature, thus significantly reducing the experimentall efforts necessary to study them.

  Of particular interest are \ccs as single photon sources when hosted in diamond. With its transparency, exceptional stability and minimal phononic interactions at room temperature the diamond lattice is an ideal host matrix for \ccs \cite{steinmetz::70,steinmetz::71}. While more than $500$ different \ccs in diamond are documented, only a small fraction has been investigated with respect to their properties as single photon sources \cite{steinmetz::58}. For an in-depth review of \ccs and their versatile applications see \cite{janine::64, janine::153}. The two arguably most prominent examples of well-studied \ccs are vacancy centers featuring nitrogen and silicon \cite{janine::165, janine::166, janine::167}.

  The silicon-vacancy (SiV) center in diamond and its properties is at the center of this thesis. The \siv has been established as an efficient single photon source at room temperature. It shows very narrow emmisiion lines with record count rates up to \SI{6.2d6}{\cps} (counts-per-second) \cite{janine::55}. The emmision of indistinguishable photons and the optical access of electronic spin states have been demonstrated \cite{janine::27, janine::15, janine::16, janine::17}, hinting at the possibility of deploying \sivs as spin-qubits.

  A \sivc is formed in a diamond lattice by substituting two cabon atoms by a silicon atom and a nearby empty lattice site respectively. The silicon atom occupies its energetically optimal position by sitting in-between two lattice sites. This is called ``split-vacancy'' configuration and induces a $D_{3d}$ symmetry with the two vacancies and the impurity alligned along the $\langle 111 \rangle$ diamond axis \cite{janine:222}, see \autoref{fig::siv_lattice}.

  \begin{figure}[htbp]
		\centering
		\testbox{\includegraphics[trim = 0 0 0 0,  clip= true, width = 0.3\textwidth]{./pics/siv_lattice.png}}
		\caption[Split-vacancy configuration for \sivs in diamond]{Crystal structure of the \siv embedded into the diamond lattice: The silicon atom (blue sphere) sits in between two vacant lattice sites (white spheres) forming a ``split-vacancy'' configuration aligned along the $\langle 111 \rangle$ crystallographic axis (yellow line). Figure reproduced with permission from \cite{janine::thesis}.}
		\label{fig::siv_lattice}
	\end{figure}


  The \siv is known to occur in two different charge states. The first is the neutral state or SiV$^0$ with a zero-phonon transition at \SI{1.31}{\eV} (\SI{946}{\nm}). It is assoctiated with a $S = 1$ ground state \cite{janine::224}. The second state is the negatively charged state SiV$^{-}$ where the \sivc recruited an additional free electron. It exhibits a zero-phonon transition at \SI{1.68}{\eV} (\SI{738}{\nm}. Its ground state has been determined as a $S = \frac{1}{2}$ state \cite{janine::222, janine::223}. Due to its outstanding brightness and the location of the \zpl in the visible range of the spectrum, this thesis focuses on the negatively charged \siv. For convenience we drop the charge distinction from now on and refer to SiV$^{-}$ centers simply as \sivs.

  \sivs have been created using \CVD in \nds and single-crystal diamond films \cite{janine::25}, see \autoref{ch::fabrication_nanodiamonds} for details. It is also possible to directly implant silicon atoms into pure diamond. High temperature annealing must then be used to animate present lattice vacancies to recombine with silicon impurities in order to forme split-vacancy \sivs \cite{janine::24,janine::223}.

  In the following sections we detail the most important luminescence properties of \sivs in diamond. For a comprehensive review we refer to \cite{janine::thesis, neu::thesis} and references therein.


  \subsection{Luminescence properties}

    The \sivc as a quasi-atomic system is capable of absorbing and emmitting light. When a ground state electron absorbs a photon of appropriate energy, it is promoted to a discrete higher-energy excited state located within the band gap of the diamond host matrix. Reversing this excitation relaxes the electron back down to the ground state while emmitting a so-called flourescent photon, accounting for the energy difference between excited and ground state. This transition is ``spin-allowed", limiting the life-times of excited states to nano-seconds and thus promoting rapid relaxation and associated fluorescence \cite{janine::229}.

    Since flourescence is directly linked to the electronic structure of the \siv, see \autoref{fig::above_resonant_excitation}. It follows that photoluminescence spectroscopy can be used to study it using a laser to optically excite the \siv. In the context of this thesis, optical above-resonant excitation is the method of choice, in particular, when used in conjunction with a confocal photoluminescence setup which will be discussed in \autoref{ch::pl_setup}. If the excitation energy exceeds the energy of the lowest excited state, electrons are promoted to higher electronic and vibrational states. Conveniently, these states relax rapidly towards the lowest exited state in non-radiative processes \cite{janine::229}. Once the lowest excited state is reached, a flouresent transition can follow. It has been shown that above-resonant excitation is feasible for excitation energies ranging from \SIrange{1.75}{2.55}{\eV} \cite{becker::34, neu::131, neu::136}. If the excitation energy is choosen too high, however, the \siv is ionized. Electrons donated to the diamond conduction band do not participate in flourescence. Ionization may be reversed if a positively charged \siv manages to capture a free electron from the conduction band. This charge state conversion is believed to be linked to flourencense intermittence, more intuitively named as blinking \sivs \cite{neu::35, neu::137}.

    \begin{figure}[htbp]
      \centering
      \testbox{\includegraphics[trim = 0 0 0 0,  clip= true, width = 0.3\textwidth]{./pics/above_resonant_excitation.png}}
      \caption[Band gap of \sivs hosted in diamond]{Simplified picture of the generous \SI{5.5}{\eV} band gap of diamond with discrete states induced by the presence of \sivs. The \siv ground state is situated \SI{2.05}{\eV} below the diamond conduction band. The lowest excited state sits \SI{1.68}{\eV} above the ground state with the \zpl transition in red connecting the two. Above-resonant optical excitation is indicated in orange. Figure curtesy of \cite{becker::thesis}.}
      \label{fig::above_resonant_excitation}
    \end{figure}

    The flourescence spectrum of \sivs have two prominent features: A narrow \zpl a broad \psb. The former is connected to flourescent photons associated with a purely electronic transition while the latter involves vibrational transitions involving electron-phonon interactions. The \psb is typically shifted to higher wavelengths with respect to the \zpl. The resulting energy deficit can be explained by phonons being created during the relaxation of higher vibrational states. A shift in the opposite direction can also be observed in rare cases if phonons are absorbed during the relaxation process \cite{becker::36}.

    The relative strength of the \zpl and the \psb is connected to the electron-phonon coupling of the excited \cc and thus to lattice vibrations. When a \cc is excited, its charge distribution changes. As a result, the equilibrium positions of all particles involved in the \cc shift leading to changes in \cc geometry. Naturally, the combined changes of charge distribution and geometry of the \cc affect the sourrounding atoms of the host lattice. Similarily, if the exited state relaxes back to the ground state, the process occurs in reverse. Thus, due to differing atomic arrangements for ground and exited states, the emmission and absorption of photons is accompanied by lattice vibrations, i.e.\ phonons \cite{janine::thesis}.

    The Huang-Rhys model allows us to discuss the electron-phonon interaction in more detail. Our discussion follows the presentation in \cite{janine::213}. The model assumes in its simplest form that vibrational modes can be modelled as oscillations of nuclei between their euqilibrium coordinates $q$ associated with the electronic states \cite{janine::thesis}. Let $\mathcal{K}_{{}^{3}A_{2}}$ and $\mathcal{K}_{{}^{3}E}$ denote the harmonic potentials of the ground and excited states respectively:

    \begin{align}
      \mathcal{K}_{{}^{3}A_{2}} & = \frac{1}{2} \Omega^2 q^2 \\
      \mathcal{K}_{{}^{3}E} & = C_{{}^{3}E} + aq + \frac{1}{2} (\Omega^2 + b)q^2 \\
       & = C_{{}^{3}E} - C_R + \frac{1}{2} (\Omega^2 + b)(q - \delta q)^2.
    \end{align}

    It follows that the vibrational modes of ground and exited states are given as harmonic states with disrete energies $\hbar \Omega (\nu + 2)$ as well as $\hbar \sqrt{\Omega + b} (\nu + 2)$ respectively, where $\nu$ denotes the occupation number and $\Omega$ the frequency.
    Further, $aq$ denotes the linear nuclear displacement of the excited state configuration with respect to the ground state equilibrium where $q = 0$.
    The quadratic term $bq^2$ refers to the vibrational frequency shoft due to a redistribution of charge between the electronic states.
    From the linear and quadratic coupling strenghts $a$ and $b$, the equilibrium displacement of the ${}^{3}E$ state is given by:

    \begin{equation}
      \delta q = \frac{-a}{\Omega^2 + b },
    \end{equation}

    while the relaxation energy reads \cite{janine::213}:

    \begin{equation}
      C_R = \frac{a^2}{2 (\Omega + b)} = \hbar S \sqrt{\Omega + b},
    \end{equation}


    where $S$ is referred to as Huang-Rhys factor.

 The harmonic potentials and states are schematically shown in figure 3.6, with the excited state potential being shifted by the equilibrium displacement δq with respect to the ground state potential.

Applying the Franck-Condon principle, the most probable electronic transition from the excited 3E state to the ground state 3A2 originates from the fundamental vibrational 3E state at δq into higher vibrational levels of the 3A2 ground state followed by non- radiative relaxation to the fundamental 3A2 vibrational level that effectively displaces the nuclei to the ground state equilibrium coordinate. The excitation from 3A2 to 3E proceeds vice versa. The Huang-Rhys factor S physically describes the mean number of vibrational quanta involved in the optical transition. For the NV center, the Huang- Rhys factor is typically S = 3.76 and the energy of the ground state vibrational mode is 􏰃Ω = 65 meV [168, 214]. The most probable transition originating from δq marks the maximum in the emission spectrum at 􏰃ω = 􏰃ωZPL −S􏰃Ω, shifted to longer wavelengths with respect to the ZPL. Other transitions are less probable and less intense. For the NV− center, we expect maximum intensity at 1.7eV (728nm). In absorption, an excitation energy of 􏰃ω = 􏰃ωZPL + S􏰃Ω = 2.19 eV (565 nm) efficiently excites the NV center, which corresponds well with the 532nm excitation wavelength widely used for off resonant excitation.
A second measure of the electronic-vibrational coupling is the Debye-Waller factor Dw, which is related to the Huang-Rhys factor via Dw = exp(−S). The Debye-Waller factor determines the ratio of photons emitted into the zero-phonon line compared to the overall emission. Due to the poor ZPL branching ratio, the Debye-Waller factor of the NV center amounts only Dw = 2 − 4% [80, 83, 125, 170].


% becker
The energy of the excited state in the case of electron phonon coupling can be written as [37]:
Ve = Ee + 1mω2q2 + aq + bq2 (2.1) 2
Here, the term Ee corresponds to the electronic energy of the excited state, whereas the term 1mω2q2 corresponds to the vibrational energy of the state
(potential approximated as being harmonic). The following two terms account for the linear electron-phonon coupling aq and the quadratic coupling bq2, with q being the displacement of the nuclei from their equilibrium positions. The linear coupling term corresponds to a constant change in the positions of the atomic nuclei between the ground and the excited state whereas the quadratic coupling describes a change in the vibrational frequencies. For the understand- ing of the PSB, the linear electron-phonon coupling has to be considered in greater detail. In Fig. 2 (b) the vibrational parabolas of an electronic ground and an excited state are shown for an arbitrary value of q. Since the excitation process (orange arrow) is very fast, it occurs vertically in terms of q, which means, that the system is excited into a higher lying vibrational state of the excited electronic state. The excitation occurs to the vibrational state with the wavefunction that has the largest overlap with the wavefunction of the initial state. Subsequently, on a larger time scale, the nuclei react on the changed charge distribution and relax in their new equilibrium positions. The excited vibrational state then relaxes into the vibrational ground state (dashed arrow) of the electronic excited state by non-radiative internal conversion processes. The following electronic relaxation process (red arrow) is again much faster than the movement of the atomic nuclei and the system will again end up in a vibrational excited state of the electronic ground state. Again, the final state which has the largest overlap with the vibrational ground state of the electronic excited state has the highest probability [38]. From Fig. 2 (b) it is easily noticeable that the value of q, which means the amount the parabolas are shifted against each other, determines the most probable vibrational state the system relaxes to. This probability is usually expressed by the so called Huang-Rhys factor S, which is given by
2 ne−S
|M0n| =S n! (2.2)
with |M0n|2 being the intensity of the transition to the nth vibrational excited state [37]. If S equals two, this means, that the most probable vibrational state the system relaxes to, is the state with n=2. This case is illustrated in Fig. 2 (b). For S approaching zero, the most probable transition becomes the zero phonon line, which means, that the ratio of the ZPL fluorescence to the total fluorescence in the spectrum is given by
IZPL = e−S. (2.3)
IZP L+P SB
The Huang-Rhys factor of the SiV centre has been determined to amount to S=0.08 up to S=0.24 [39, 40, 41]. This means that the major part of the flu- orescence of the SiV centre is concentrated in its zero phonon line making it a very narrowband single photon source (even for S=0.24 only around 23% of the fluorescence would be emitted by the PSB).
A transition at 1.681eV in the spectrum has been assigned to the negatively charged SiV centre [42, 43] whereas a transition at 1.31 eV has been attributed to the neutral charge state [44]. However, since the fluorescence of the neutral SiV charge state is usually relatively weak the interest focusses on the bright 1.681 eV transition of the negatively charged centre. If SiV centres are cooled below approximately 110 K, the ZPL splitts into several fine structure compo- nents [45]. As seen in Fig. 3 (a), in large ensembles of SiV centres, up to twelve different lines can be observed with intensities corresponding to the natural abundances of the three stable isotopes of silicon. Each isotope contributes with four lines to the total spectrum. The lines are labelled with A1 to D1 etc.
starting with the line of highest energy. As shown in Fig. 3 (b), in a spectrum of a single SiV centre this four-line fine structure is more obvious. The distri- bution of the fluorescence intensity over the four lines shown fo this emitter is characteristic for a SiV centre in bulk diamond exhibiting only a small local crystal strain. This rather unique pattern unveils, together with temperature- dependent photoluminescence measurements, that the electronic structure of the centre has to contain two levels in the ground and the excited state, with a splitting of around 0.02 meV and 1.07 meV, respectively[43]. In the following chapters, this splitting pattern will be a starting point to get more detailed information about the electronic structure as well as the symmetry properties of the negatively charged SiV centre.






    \subsection{Excitation}

      To observe single photons being emmitted from \sivs hosted in diamond, color centers first need to be excited. In the context of this thesis, optical above-resonant excitation is the method of choice, in particular, when used in conjunction with a confocal photoluminescence setup, see \autoref{ch::pl_setup}.

      It is known that a above-resonant excitation of \sivs is feasible for excitation energies ranging from \SIrange{1.75}{2.55}{\eV} \cite{becker::34}.

      For \sivs it is known that above-resonant excitation promotes electrons to states below the edge of the conduction band of the diamond host as long as excitation energies do not exceed \SI{2.05}{\eV} \cite{becker::35}.

      Using an exciation laser with a wavelength above the flouresent wavelength of \sivs causes electrons to be promoted from the ground state to excited states with energy levels exceeding the energy of the lowest excited state. Conveniently, the dissipative nature of the \sivc is such that electrons in higher exited states quickly relax down to the lowest excited state in a non-radiative process. From there electrons may relax further down to the ground state, emmiting observable single flourescent photons. \autoref{fig::above_resonant_excitation} illustrates the process. We remark that possible shelving states exhibiting both radiative and non-radiative dissipation are believed to be involved in the \siv relaxation processes \cite{janine::229}. While we omit a discussion of these states and their implications here, the interested reader is referred to \cite{janine::thesis} and references therein for a detailed description of the electronic (fine)structure of \sivs.

      \begin{figure}[htbp]
    		\centering
    		\testbox{\includegraphics[trim = 0 0 0 0,  clip= true, width = 0.3\textwidth]{./pics/above_resonant_excitation.png}}
    		\caption[Above-resonant excitation of \sivs hosted in diamond]{Simplified picture of above-resonant optical excitation of \sivs in nanodiamonds \cite{neu::thesis}.}
    		\label{fig::above_resonant_excitation}
    	\end{figure}

      To verify wheter a \siv can be optically excited, transitions of the \siv can be observed using direct absorption \cite{neu::117, neu::118}. Alternatively photoluminescence exitation spectroscopy may be used to investigate the fluourescence intensity of \ccs as a function of different excitation wavelengths. Using this method it has been shown that \sivs can be excited (with variying effciency) over a broad spectral range between \SIrange{1.75}{2.55}{\eV} \cite{neu::131, neu::136}.

      In this work we will use the following excitation energies\todo[fancyline]{Fill in wavelengths used in this thesis and explain why they are used.}.


    \subsection{Zero phonon line}

      % neu
      Generally, the purely electronic transition of a color center, the zero-phonon-line (ZPL), is the spectrally narrowest transition of the color center. In contrast, vibra
      tion assisted electronic transitions (vibronic sidebands) are much broader. Thus, to enable low bandwidth single photon emission, a color center’s fluorescence should concentrate into a narrow ZPL. The SiV center is a promising candidate for such a low bandwidth emitter as discussed in the following sections.
      Previous reports in the literature often reflect the luminescence properties of (large) ensembles of SiV centers. Feng et al. [139] report a linewidth of 15meV (6.5nm) for an ensemble of SiV centers in polycrystalline diamond (PCD). Gorokhovsky et al. [112] specify 13.6 meV (6 nm) in the same material system. More recently, Vlasov et al. [132] find a width of 7 nm for SiV centers in PCD and 8 nm for ultrananocrystalline diamond. In highly stressed low quality PCD, the ZPL of the SiV center might be split and broadened to span the wavelength range of 733 − 745 nm [96]. Thus, several reports indicate that the ZPL of the SiV centers may be sufficiently narrow to enable low bandwidth single photon emission. The spread of linewidths observed in these reports indicates that the linewidth of SiV ensembles is influenced by inhomogeneous broadening. The significant influence of inhomogeneous broadening becomes especially visible in low temperature spectra, where the homogenous broadening due to phonons that dominates at room temper- ature is strongly reduced and only the linewidth due to inhomogeneous broadening remains. A detailed discussion of the line broadening mechanisms for the ZPL is presented in Sec. 7.2.1. Feng et al. [139] measure a linewidth of 10meV (4.4nm) at 10K. Clark et al. [140] find a width of 16.8meV (7.4nm) at 77K, with a re- duction to 8.4meV (3.7nm) after an HPHT annealing step. The major source for the inhomogeneous broadening is mechanical stress in the diamond [96]. A detailed discussion of the sources of stress in the samples employed in this work can be found in Sec. 5.2.1.
      Only two reports of the linewidth of single centers exist: Wang et al. report a ZPL linewidth of 5 nm [37] and 6 ± 1 nm [74] at room temperature. We point out that in the work of Wang [74] also centers with significantly lower linewidth have been observed: A center with an emission wavelength of 736.8 nm and 1.3 nm ZPL width is not identified as an SiV center due to the short emission wavelength. Nevertheless, this blue shift of the ZPL might be due to the crystalline environment in the vicinity of the center: As mentioned above, reports in the literature clearly indicate inhomogeneous broadening of the ZPLs of ensembles of SiV centers in the nm range. Thus, one expects line positions spread over the inhomogeneous linewidth when observing single emitters and a ZPL observed at 736.8 nm might be due to an SiV center.
      At low temperature, in diamond with a low inhomogeneous broadening of the ZPL, a four line fine structure of the ZPL transition has been observed [111,140]. This specific line pattern might be considered as the ’spectral fingerprint’ of the SiV center and might be used to prove the identification of single color centers as SiV centers. A detailed discussion of the pattern as well as the associated level scheme is performed in Sec. 7.1. Together with a narrowing of the line upon cooling and the occurrence of the four line pattern, the SiV ZPL blue shifts 1.2-1.4nm upon cooling [112,139].

    \subsection{Vibrational side-bands}

      % neu
      The linear electron-phonon coupling of the SiV center will be discussed in the following, while further discussion of quadratic electron-phonon coupling is shifted to section 7.2.2, where it is used to interpret temperature dependent spectra of single SiV centers. Vibronic sidebands in emission mostly occur as transitions from the vibrational ground state (n′ = 0) of the excited electronic state to higher vibrational states (n > 0) of the electronic ground state (for the level scheme see Fig. 1.6). Consequently, these lines are red shifted compared to the ZPL in luminescence. The red shift gives the phonon energy of the corresponding phonon mode if n = 1 (one-phonon sideband). For higher order sidebands (n > 1), the energy is a multiple of the phonon energy. In absorption, sidebands are blue shifted compared to the ZPL. Here, absorption mostly takes places from the vibrational ground state (n = 0, highest population) in the electronic ground state to higher vibrational states (n′ > 1) in the excited electronic state. The sideband spectrum together with the ZPL is characteristic for a specific color center [25], assisting to identify the emitting centers. The vibronic sidebands of the SiV center are weak compared to the ZPL even at room temperature. This indicates a weak linear electron-phonon coupling and renders SiV centers especially suitable as low bandwidth emitters. The linear electron- phonon coupling is measured either by the Debye-Waller factor (DW) or the Huang- Rhys factor (S). The Debye-Waller factor is defined as the integrated luminescence intensity of the ZPL Izpl divided by the integrated luminescence intensity of the
      color center Itot [27]. The Huang-Rhys factor S is defined by Izpl = exp(−S) [84]. Itot
      The Huang-Rhys factor can be interpreted as an indication of the most probable (vibration assisted) transition (discussion following Ref. [84]). Figure 1.6 indicates the situation for a Huang-Rhys factor S close to zero and for S = 2. For a very low Huang-Rhys factor, the most probable transition is the ZPL, it dominates the luminescence spectrum. For a Huang-Rhys factor of S = 2, the most probable absorption transition brings the color center to the vibrational state with n′ = S = 2 phonons in the excited electronic state. This transitions marks the maximum of the absorption band. Subsequently, the color center relaxes the excess energy of n′ = S phonons in the excited state. The most probable luminescence transition then brings the color center to the vibrational state with n = S phonons in the ground state. This transition marks the maximum of the emission band. The intensity |M0n|2 of the vibronic sideband involving n phonons is given by [94]
      2 n e−S
      |M0n|=S n!. (1.5)
      In the literature, measurements on SiV center ensembles have been used to determine the Huang-Rhys factor S: Rossi et al. [136] report S = 0.1 deduced from lumines- cence measurements at room temperature using 457nm excitation. Gorokhovsky et al. [112] report S = 0.08 also deduced from luminescence measurements at 9K using 515 nm excitation. Collins et al. [118] report S = 0.24 ± 0.02 in absorption at room temperature. These experiments have been performed using polycrystalline CVD diamond grown on Si substrates. We point out that different spectral ranges have been investigated in the publications mentioned above: Ref. [136] discusses spectra ranging down to 1.4 eV, while Ref. [112] gives spectra only down to 1.59 eV for 515 nm excitation. In Ref. [118], absorption spectra with energies up to 1.78 eV are used. Due to the restricted spectral ranges investigated, the question whether all possible sidebands have been taken into account might be raised (for a discus- sion of sideband energies of the SiV center see below). Additionally, in Ref. [136] a background substraction is performed before S is calculated from the luminescence spectrum, while this is not discussed in Refs. [112] and [118]. Furthermore, the authors of Refs. [112,118,136] do not discuss whether the employed data has been corrected for a varying detection efficiency of the experimental setup throughout the investigated spectral range. Considering these experimental issues, it is not clear which of the values most closely describes the linear electron-phonon coupling of the SiV center and how the values compare to each other.
      Assuming a Huang-Rhys factor of S = 0.24, the relative intensities compared to the ZPL are 24\% for the one-phonon sidebands and only 2.8\% for the two-phonon sideband. Thus, the ZPL dominates the spectrum. Additionally, one would not expect to observe sidebands related to two-phonon processes for SiV centers. These findings are in accordance with reports in the literature that in general color centers involving heavier impurities tend to exhibit low linear electron-phonon coupling [25]. In contrast, for nitrogen vacancy centers a Huang-Rhys factor of S = 3.73 [94] leads to nearly undetectable ZPL emission at room temperature and very strong sideband emission.
      Figure 1.6 displays an extremely simplified picture of electron-phonon coupling assuming that only one vibrational mode couples to the color center. Additionally, the nature of the mode responsible for the sideband transitions has not been dis- cussed yet. In a solid state host, different types of vibrational modes can lead to sideband emission:
      X Lattice modes: These modes correspond to lattice vibrations of the undis- turbed diamond lattice.
      X Local and quasi-local modes: These modes are specific of the defect. They represent vibrations involving the defect and its neighboring carbon atoms.
      Coupling to lattice modes is governed by the phonon density of states of the diamond lattice. The density of states has been calculated and measured in the literature, e.g., in Refs. [25, 143–145]. Here, electronic transitions predominantly couple to phonons with wave vectors at the high symmetry points of the Brillouin zone, the so called critical points [25,139]. This has two reasons: First, these phonons have the shortest wavelengths. Thus, they can induce the strongest changes in interatomic distances, as their wavelengths are comparable to the spatial dimension of the color centers. This enables efficient coupling [25]. Second, the phonon density of states peaks at these points [25,139], as the phonon dispersion relations displays a van- ishing slope [144,146]. A list of the phonon energies corresponding to these critical points is summarized in Ref. [147] and is displayed in Tab. 1.1. Electronic transitions predominantly couple to optical phonons as the anti-phase movement of the lattice atoms in these modes evidently induces strong interatomic distance changes. For acoustic modes, only short wave acoustic modes induce significant interatomic dis- tance changes and thus couple efficiently. The phonon density of states in diamond has a sharp high energy cut off at around 165meV and diminishes strongly be- low approx. 70meV [25,143–145]. Therefore, all sideband features shifted less than 70 meV or more than 165 meV cannot be induced by modes of the diamond lattice: Features with high energy shift may arise from multi-phonon processes. However, for SiV centers, multi-phonon processes are very improbable and the features most probably arise from local modes of the color center. In contrast, the features with a low energy shift are fully attributed to local modes of the color center.
      Quasi-local and local modes arise solely due to the presence of the impurity. If the energy of these modes lies within the energy range of the lattice phonon density of states, the modes are referred to as quasi-local. If the energy is > 165meV or < 70 meV they are termed local modes. The vibrational frequencies of these modes are determined by the masses of the impurities as well as the interatomic bonding forces that act as springs for the vibration [25]. Brout and Vischer introduced a simplified model to determine the quasi-local mode frequencies ωQL and their resonance width ∆ωQL for heavy impurities assuming that these impurities do not change the interatomic forces significantly [25,148].

      √
      ω=ω MC ∆ω=πω MC (1.6) QL D 3(nMI −MC) QL 6 D3(nMI −MC)
      Here, MC and MI are the masses of the carbon atom and the impurity respectively. n denotes the number of impurity atoms participating in the vibration. ωD is the Debye frequency in diamond, which corresponds to 150 meV. For an impurity incor- porating one Si atom, one would thus expect a quasi-local mode at 75 meV. Indeed, sidebands with a comparable energy shift at approx. 80meV have been observed (see Tab. 1.1). More advanced calculations using estimated force constants for lat- tice and defect were performed in Ref. [143]. The results for a Si atom sitting in a ’split-vacancy’ site are summarized in Tab. 1.1, indicating the existence of several high energy local modes.
      Vibronic sidebands of SiV centers have been examined in several publications. The results are summarized in Tab. 1.1. Sittas et al. [115] examined the sideband structure in Si-doped HPHT diamonds. In low temperature experiments, they find a significant spectral narrowing of the three highest energy features corresponding to emission at 776 nm, 797 nm and 812 nm. Thus, they attribute these lines to purely electronic rather than vibronic transitions. This indicates that discrimination of vibronic sidebands and electronic transitions is not always clear as for defects involving heavy impurities the linewidths of sideband features due to local modes can be comparable to the ZPL linewidth [25].
      There are several indications in the literature that vibronic sidebands and, there- fore, the linear electron-phonon coupling for SiV centers significantly depends on the local environment. Sternschulte et al. find a sideband feature shifted by 166meV only in some positions on a homoepitaxial CVD diamond [111]. Gorokhovsky et al. report that in polycrystalline diamond vibronic sideband features could not be resolved under non-resonant 514 nm laser excitation [112]. However, upon resonant excitation with 737 nm, a distinct sideband structure evolves. The authors explain that observation in terms of the excitation of only a sub-ensemble featuring a ZPL at the excitation laser wavelength. This sub-ensemble within the inhomogeneously broadened ensemble displays defined vibronic sidebands. These sidebands are spec- trally washed-out if all inhomogeneously broadened SiV centers are excited and the differing sideband spectra add up. This observation supports the assumption of strongly environment dependent electron-phonon coupling. Thus consequently, for single emitters, one would also expect to observe varying vibronic sidebands for individual emitters.
      Additionally, the authors of Ref. [112] report that if the excitation wavelength is tuned to longer wavelengths, thus selecting another sub-ensemble, the vibronic sideband structure changes: The most prominent vibronic sideband now displays a smaller energy shift with respect to the ZPL. Thus, the energy of the corresponding phonon mode has to be reduced for the sub-ensemble selected with longer wavelength laser excitation. In contrast, the shifting behavior for a higher energy sideband shows no clear trend. Thus, different vibronic sidebands show a deviant behavior under environmental changes that shift the ZPL. Furthermore, simultaneously the width of the phonon sidebands changes [112].
      As displayed in Tab. 1.1, vibronic sidebands identified as local modes as well as sidebands due to lattice phonons [147] have been reported for SiV centers. The reports in the literature display a spread in the observed positions and also in the number of features observed, again indicating a dependence on the environment. If the diamond lattice is stressed, lattice phonon energies change. Additionally, energetically degenerate phonon modes split up [149]. Shifting rates for the optical zone center phonon (165.2 meV = 1332.4 cm−1) amount to [149]
      ∆hydro = 0.40±0.02meV ∆[111] = 0.27±0.02meV ∆[100] = 0.09±0.01meV GPa GPa GPa
      for hydrostatic pressure and compressive uniaxial stress along [111] or [100] direction. The optical zone center phonon is the only phonon were a direct observation of the shift due to stress is possible as it can be detected via Raman scattering of light. For local modes, however, no similar direct measurements of the phonon energies is possible. Thus, it is challenging to discriminate between changes in the electron- phonon coupling of a color center or changes in the mode frequencies.
