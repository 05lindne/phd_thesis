%!TEX root = ../../main.tex


%advice: in linux, put new style files into something like /usr/share/texmf/tex/latex and then run ``texhash''


% \usepackage{newclude}	% for the option \include*{...}, to prevent pagebreak between consecutive sections saved in two different files. (reimplementation of \include and \includeonly -> http://www.ctan.org/pkg/newclude)
\usepackage[pagebackref=true,plainpages=false,pdfpagelabels,breaklinks=true]{hyperref}
\usepackage[T1]{fontenc}
\usepackage{lmodern}
\usepackage{caption}
\usepackage{lineno}	% allows line numbering: \linenumbers
\usepackage{setspace}	% allows line spacing: \singlespacing,\onehalfspacing,\doublespacing
\usepackage{graphicx}
% \usepackage{minipage}
\usepackage{color}
\usepackage{makeidx}
\usepackage{fancyhdr}	%options for headers and footers
\usepackage[ngerman,english]{babel}
\usepackage{mathtools}%extention to amsmath; loads amsmath;
\usepackage{xspace}% when used in newcommand, space is only inserted, if not followed by dot, comma... Must be written at the LAST spot within the braces of a newcommand!!!
\usepackage{parskip}
\usepackage[singlelinecheck=false, aboveskip=-3pt]{subcaption}% use this package if subfigures are wanted, contains newest implementation of subfigures
\usepackage{cleveref}%supports cref to reference multiple things in a row and to automatically write into the text what object is referenced to(http://get-software.net/macros/latex/contrib/cleveref/cleveref.pdf)
\usepackage[plain]{fancyref}%usage: \autoref{type:title} or \autoref{type:title}, gets information about kind of referenced stuff and writes name&number
\usepackage{textcomp}
\usepackage{gensymb}%for units
\usepackage{siunitx}
\DeclareSIUnit\inch{in}
\usepackage[autostyle,english=american]{csquotes}
\usepackage{chemformula}
\usepackage{physics}
\usepackage{mdframed} % Add easy frames to paragraphs

% \usepackage{showlabels}
% \usepackage{showframe}

\usepackage{booktabs}% to use \toprule, \bottomrule in table
\usepackage{backref}
\usepackage[inline]{enumitem} % inline option which implements inline versions of the standard lists using starred versions of the basic list environments.
\usepackage{float}

\usepackage[showframe,headheight=14pt, margin=2.2cm,includehead,includefoot]{geometry}% Set margin sizes, put in showframe to show the frames on document
\usepackage[textwidth=0.75in]{todonotes}
\presetkeys{todonotes}{fancyline}{}

\geometry{bindingoffset=1cm}


\hypersetup{%
  pdftitle=PhD Thesis,
  pdfauthor=Sarah Lindner,
  colorlinks=false,
  pdfborderstyle={/S/U/W 3}}

% \setlength{\headheight}{13.7pt}%make the header size higher (e.g. if superscripts used in header)

\DeclareGraphicsExtensions{.eps}

% \pagestyle{fancy}

%\usepackage[nomarkers,nofiglist,figuresonly]{endfloat} %put all figures to the end of document

% \reversemarginpar

\usepackage{blindtext}  %generates lorem impsum latin blind text
\usepackage[nodayofweek,level]{datetime}

\usepackage{epigraph}
\usepackage{etoolbox}

\usepackage{tikz}
\usetikzlibrary{spy}

\setlength\epigraphwidth{\linewidth}
\setlength\epigraphrule{0pt}
\renewcommand{\epigraphflush}{center}
