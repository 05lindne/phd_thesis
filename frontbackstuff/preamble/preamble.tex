%!TEX root = ../../main.tex

\usepackage{scrhack}			%supresses a useless KOMA script warning
\KOMAoptions{paper=a4,twoside,fontsize=12pt,BCOR=12mm}
\KOMAoptions{headinclude=false,footinclude=false}
\KOMAoptions{headsepline=true}
\KOMAoptions{open=right}
\addtokomafont{pageheadfoot}{\linespread{1}\selectfont}
\usepackage[utf8]{inputenc}

% \usepackage{newclude}	% for the option \include*{...}, to prevent pagebreak between consecutive sections saved in two different files. (reimplementation of \include and \includeonly -> http://www.ctan.org/pkg/newclude)
\usepackage[pagebackref=false,plainpages=false,pdfpagelabels,breaklinks=true]{hyperref}
\usepackage{etex}
\usepackage[T1]{fontenc}
\usepackage[format=plain,labelfont=bf]{caption}
\usepackage{lineno}	% allows line numbering: \linenumbers
\usepackage{setspace}	% allows line spacing: \singlespacing,\onehalfspacing,\doublespacing
\usepackage{graphicx}
% \usepackage{minipage}
\usepackage{color}
\usepackage{makeidx}
\usepackage{fancyhdr}	%options for headers and footers
\usepackage[ngerman,english]{babel}
\usepackage{mathtools}%extention to amsmath; loads amsmath;
\usepackage{xspace}% when used in newcommand, space is only inserted, if not followed by dot, comma... Must be written at the LAST spot within the braces of a newcommand!!!
\usepackage{parskip}
\usepackage[singlelinecheck=false, aboveskip=-3pt]{subcaption}% use this package if subfigures are wanted, contains newest implementation of subfigures
\usepackage{cleveref}%supports cref to reference multiple things in a row and to automatically write into the text what object is referenced to(http://get-software.net/macros/latex/contrib/cleveref/cleveref.pdf)
\usepackage[plain]{fancyref}%usage: \Fref{type:title} or \Fref{type:title}, gets information about kind of referenced stuff and writes name&number
\fancyrefchangeprefix{\fancyrefchaplabelprefix}{ch}
\usepackage{textcomp}
\usepackage{gensymb}%for units
\usepackage{siunitx}
\DeclareSIUnit\inch{in}
\usepackage[autostyle,english=american]{csquotes}
\usepackage{chemformula}
\usepackage{physics}
\usepackage{mdframed} % Add easy frames to paragraphs

% \usepackage{showlabels}
% \usepackage{showframe}

\usepackage{booktabs}% to use \toprule, \bottomrule in table
\usepackage[inline]{enumitem} % inline option which implements inline versions of the standard lists using starred versions of the basic list environments.
\usepackage{float}

\usepackage[showframe,headheight=15pt, margin=2.2cm,includehead,includefoot]{geometry}% Set margin sizes, put in showframe to show the frames on document
\usepackage[textwidth=0.75in, disable]{todonotes}
\presetkeys{todonotes}{fancyline}{}

\geometry{bindingoffset=1cm}

  \hypersetup{
    colorlinks=false, linktocpage=true, pdfstartpage=3, pdfstartview=FitV,%
    breaklinks=true, pdfpagemode=UseNone, pageanchor=true, pdfpagemode=UseOutlines,%
    plainpages=false, bookmarksnumbered, bookmarksopen=true, bookmarksopenlevel=1,%
    hypertexnames=true, pdfhighlight=/O,%
    pdfauthor={Sarah Lindner }, pdftitle={Towards~Integrated~Single-Photon~Sources~Using~the~Silicon~Vacancy~Center~in~Nanodiamonds}, pdfcreator={pdflatex},%
  pdfborderstyle={/S/U/W 3},%
  }

% \setlength{\headheight}{13.7pt}%make the header size higher (e.g. if superscripts used in header)

\DeclareGraphicsExtensions{.eps}

% \pagestyle{fancy}

%\usepackage[nomarkers,nofiglist,figuresonly]{endfloat} %put all figures to the end of document

% \reversemarginpar


\PassOptionsToPackage{
  backend=biber,
  language=auto,
  % style=ieee, %
  %style=numeric-comp,%
  % style=authoryear-comp, % Author 1999, 2010
  sorting=nyt, % name, year, title
  maxbibnames=3, % default: 3, et al.
  % backref=true,%
  natbib=true, % natbib compatibility mode (\citep and \citet still work)
  url=false,
  isbn=false,
  doi=false,
  eprint=false,
  %location=false,
  %series=false,
  %issue_date=false,
  %edition=false,
  %language=false
}{biblatex}
\usepackage{biblatex}
\AtEveryBibitem{\clearlist{language}} % clears language
\AtEveryBibitem{\clearfield{note}}    % clears notes
\AtEveryBibitem{\clearfield{doi}}   
\AtEveryBibitem{\clearfield{location}}   
\AtEveryBibitem{\clearfield{edition}}   
\AtEveryBibitem{\clearfield{series}}   
\AtEveryBibitem{\clearfield{address}}   
\AtEveryBibitem{\clearfield{month}} 
\AtEveryBibitem{\clearfield{isbn}} 


\usepackage{blindtext}  %generates lorem impsum latin blind text
\usepackage[nodayofweek,level]{datetime}

\usepackage{epigraph}
\usepackage{etoolbox}

%table of contents options
\usepackage{tocbibind}

%beautify font
\usepackage{lmodern}
\usepackage[activate={true,nocompatibility},final,tracking=true,kerning=true,spacing=true,factor=1100,stretch=10,shrink=10]{microtype}
\SetTracking{encoding={*}, shape=sc}{10}
\usepackage{titlesec}

\usepackage{tikz}
\usetikzlibrary{spy}

\setlength\epigraphwidth{\linewidth}
\setlength\epigraphrule{0pt}
\renewcommand{\epigraphflush}{center}

% adding relevant code to use "subfig" as a valid prefix to make \Fref{subfig::*} identical to \Fref{fig::*}
\newcommand*{\fancyrefsubfiglabelprefix}{subfig}
\newcommand*{\subfigname}{Figure}
\newcommand*{\Frefsubfigname}{Figure}
\frefformat{plain}{\fancyrefsubfiglabelprefix}{%
\Frefsubfigname\fancyrefdefaultspacing#1\@\xspace
}%
\Frefformat{plain}{\fancyrefsubfiglabelprefix}{%
\subfigname\fancyrefdefaultspacing#1\@\xspace
}%

\newcommand*{\fancyrefsubseclabelprefix}{subsec}
\newcommand*{\subsecname}{Section}
\newcommand*{\Frefsubsecname}{Section}
\frefformat{plain}{\fancyrefsubseclabelprefix}{%
\Frefsubsecname\fancyrefdefaultspacing#1\@\xspace
}%
\Frefformat{plain}{\fancyrefsubseclabelprefix}{%
\subsecname\fancyrefdefaultspacing#1\@\xspace
}%

\newcommand*{\fancyrefsubsubseclabelprefix}{subsubsec}
\newcommand*{\subsubsecname}{Section}
\newcommand*{\Frefsubsubsecname}{Section}
\frefformat{plain}{\fancyrefsubsubseclabelprefix}{%
\Frefsubsubsecname\fancyrefdefaultspacing#1\@\xspace
}%
\Frefformat{plain}{\fancyrefsubsubseclabelprefix}{%
\subsubsecname\fancyrefdefaultspacing#1\@\xspace
}%

