%!TEX root = ../main.tex

\chapter{Experimental Setup}	\label{ch::experimental_setup}
\chaptermark{Experimental Setup}

	I this chapter, the experimental setup which was used for the majority of the experiments described in this thesis will be introduced. 
	After a short overview, it will be described in further detail.

	\section{Intro Confocal Setup}

	The key measurements of this thesis are fluorescence measurements of \sivs in nanodiamonds.
	For this aim, a home built confocal setup is used.
	The setup serves to perform a series of measurement of fluorescence light: scanning the sample to find \sivs, recording luminescence spectra of the aforementioned, determine the saturation count rate and determine whether the emitter in question is a single emitter by performing photon autocorrelation measurements.
	The two key components for these measurements are 

	\begin{itemize}
		\item A Hanbury-Brown and Twiss setup to investigate the single photon character. It is built up of two avalanche photo diodes (APDs) which also serve to scan the sample in order to find emitters on the sample surface; and to perform saturation measurements.
		\item A spectrometer to investigate the spectral properties.
	\end{itemize}
	
	A slightly modified version of the setup is used to perform measurements of the Raman emission of the diamond host material.

	\section{Details Confocal Setup}

	\section{Raman Measurements}
	\section{Nanomanipulator}

