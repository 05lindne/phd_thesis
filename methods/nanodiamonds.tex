%!TEX root = ../main.tex

\chapter{Fabrication of Nanodiamonds}	\label{ch::fabrication_nanodiamonds}
\chaptermark{Fabrication of Nanodiamonds}

	"Diamond forms under high temperature and pressure conditions that exist only about 100 miles beneath the earth’s surface." (Homepage of the Gemological Institute of America Inc.)
	While this statement is true for natural gem diamonds, various methods exist to synthetically produce diamond for applications in industry and research. 
	In this chapter, different fabrication methods of nanodiamonds are explained.
	The first two procedures described are the \hpht method and the \cvd are described.
	These are the most commonly used fabrication methods for laboratory-produced diamonds. 
	The \hpht (\HPHT) process is similar to the natural growth process within earth and is widly used to synthetically produce diamonds for industry.
	Many measurements which are subject in this thesis are carried out on nanodiamonds produced with a \CVD process. 
	This process is also the prerequisite for the third method mentioned, the wet-milling in a vibrational mill.
	The main focus of this thesis is on wet-milling nanodiamonds, which is a novel technique using \cvd diamond as starting material. 
	It has to be stressed, that in contrast to the other methods described in this chapter, the wet-milling process is not a process to produce diamond itself, rather it serves to crush a bigger diamond into pieces of  a desired size.
	At the end, the methods of producing diamonds via detonation processes and sonicating graphite powders are briefly described.
	As diamonds produced with this process is not scope of this thesis, they will only be shortly introduced to complete the list.

	

	\section{High-Pressure High-Temperature Diamond}

	The \HPHT process was the first process with which diamond was sucessfully synthesiced (in 1879).
	Today, it is still widely used due to the relativly cheap production costs\cite{wikiSyntheticDiamond}.

	\section{Chemical Vapor Deposition Diamond}

	In constrast to 	

	\section{Wet-Milled Nanodiamonds}
	\section{Detonation and Sonification Processes}